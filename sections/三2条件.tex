项目申请人及所在团队隶属于厦门大学航空航天学院,学校与学院
可为本项目提供稳定的科研办公环境及必要的配套保障。
与此同时,申请人与德国慕尼黑工业大学空气动力学与流体力学
研究所保持长期学术交流与合作基础,并围绕高速氢预冷发动机
热管理需求下的气液两相相变(以沸腾相变为代表场景)及其
传热耦合的高可信数值模拟方法,已持续开展沟通交流与技术对接。
上述校内外条件有助于本项目在算法研发过程中开展对照评估
与交叉验证,为研究工作的连续推进与成果可靠性提供支撑。

\subsection{软件条件}
本项目的软件实现与系统性验证依托稳定的工程化计算平台条件。
申请人长期参与并持续维护开源多物理场SPH仿真平台SPHinXsys。
该平台由德国慕尼黑工业大学空气动力学与流体力学研究所相关
团队发起并发展,面向多物理强耦合问题形成了统一计算框架与
开源生态。
申请人此前在该研究所工作期间参与平台研发与工程化建设,回国
后仍持续参与平台的开发维护与协同迭代,在开源工程环境下形成了
规范化开发、跨平台部署与版本管理能力,可为本项目算法实现与
结果复现提供可靠保障。

围绕本项目拟开展的研究内容,SPHinXsys在软件层面提供直接支撑:
平台具备多相界面与复杂边界处理、时间推进与粒子管理等通用能力,
可承载本项目面向界面耦合的离散算子与更新流程实现;平台提供
传热相关模块及耦合接口,可支持将本项目提出的相变闭合与能量
守恒更新流程落地为可复现实现,并在标准化算例中开展可信性评估;
同时平台支持 CPU/GPU 并行计算等工程能力,为本项目在多工况、
多参数条件下开展稳定域标定、方案对比与系统性验证提供必要的计算支撑。
平台总体框架以及与本项目相关的功能模块对应关系见图\ref{SPHinXsys}。

%
\begin{figure}[h!]
	\centering %图片居中
	\includegraphics[width=14.7cm]{figures/软件框架}
	\captionsetup{justification=centering} %图题居中
	\caption{开源平台SPHinXsys的模块化架构及对本项目的支撑条件概览}
	\label{SPHinXsys}
\end{figure}
%

\subsection{硬件条件}
本项目依托厦门大学高性能计算中心与课题组自有算力资源,可充分满足
大规模三维仿真任务的需求。该并行计算平台由50台IBM计算刀片组成,
使用56GB带宽的Infiniband无阻交换技术互联,拥有Intel最新技术的
Xeon E5-2690V2(3.0Ghz,10Core)计算专用CPU100颗,计算专用内存
2560G,高速计算存储72TB,总体浮点计算能力达到28Tflops。
申请人所在团队,拥有计算机集群两组(26×32CPU,40×32CPU),
以及10台32核64G内存的高性能工作站。

