申请人及所在团队隶属于厦门大学航空航天学院,学校与学院可为
本项目提供稳定的科研办公环境与必要的配套条件。
申请人与德国慕尼黑工业大学空气动力学与流体力学研究所保持
长期学术交流与合作,围绕高速氢涡轮发动机热管理需求下的气液
相变两相流(以液氢预冷为代表场景)及其传热耦合的数值建模
与算法实现,已建立持续技术讨论与对接机制。
上述校内外条件可有效支撑本项目算法开发与对照验证的持续开展。

\subsection{软件条件}

本项目已具备稳定的软件平台支撑,算法实现与系统性验证依托开源
多物理场SPH仿真软件平台SPHinXsys。
该平台由德国慕尼黑工业大学空气动力学与流体力学研究所相关团队
开发并持续更新,面向多物理强耦合问题已形成统一计算框架与开源生态。
申请人作为该软件平台主要开发人员之一,长期承担核心算法开发与
工程化维护工作,为平台在本项目周期内的持续迭代、版本管理与结果
复现提供保障。
平台具备较完善的建模-求解-后处理软件链路,覆盖几何与粒子生成、
SPH通用计算内核等关键环节。

围绕本项目拟开展的研究内容,SPHinXsys可在软件层面提供直接支撑:
平台已具备多相界面与复杂边界处理、时间推进与粒子管理等通用能力,
可承载本项目界面耦合离散算子及更新流程的实现;平台提供传热相关
模块与耦合接口,可支持相变闭合与能量守恒更新流程的集成与复现,
并在标准化基准算例中开展对照验证与误差评估;
同时平台支持CPU/GPU并行计算,可满足多工况、多参数条件下的方案
对比与系统性验证需求。
平台总体框架及与本项目相关的功能模块对应关系如图\ref{SPHinXsys}
所示。

%
\begin{figure}[h!]
	\centering %图片居中
	\includegraphics[width=14.7cm]{figures/软件框架}
	\captionsetup{justification=centering} %图题居中
	\caption{开源平台SPHinXsys的模块化架构及对本项目的支撑条件概览}
	\label{SPHinXsys}
\end{figure}
%

\subsection{硬件条件}
本项目依托厦门大学高性能计算中心与团队自有算力资源,能够
满足大规模三维仿真与多工况对照评估需求。
学校并行计算平台由50台IBM计算刀片组成,使用56GB带宽的
Infiniband无阻交换技术互联,拥有Intel最新技术的Xeon
E5-2690V2(3.0Ghz,10Core)计算专用CPU100颗,计算专用内存
2560G,高速计算存储72TB,总体浮点计算能力达到28Tflops。
申请人所在团队另拥有两组计算机集群(26$\times$32 CPU,
40$\times$32 CPU)及10台32核、64GB内存的高性能工作站,
可用于日常开发调试、回归测试与批量算例计算,为本项目持续
迭代与阶段性验证提供算力保障。

