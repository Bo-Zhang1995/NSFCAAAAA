% !TeX spellcheck = en_US
\section{立项背景及研究意义}
%
面向深冷热管理与航天热防护等高热流密度工况,沸腾冷却、薄膜传热
与液滴蒸发等界面换热过程普遍呈现为气液界面流动、传热与相变的强
耦合。
在该耦合过程中,界面剧烈形变与拓扑演化、物性随温度跨数量级跃变,
以及潜热快速交换所诱发的强温度梯度,共同驱动非平衡条件下复杂的
界面动力学行为,并直接影响典型装备系统的热管理效率、运行安全与
性能边界。
现有工程设计多依赖经验关联式外推,其适用范围与不确定性边界难以
界定,直接制约安全裕度与性能边界判定。
\hl{因此,深入揭示气液界面流动、传热与相变的耦合机制,是理解复
杂工况多相传热规律并支撑工程设计的关键基础。}
受制于成本、安全性与测量分辨率等因素,实验难以在强温差与界面快
速演化并存条件下实现全过程、多物理量的同步观测;
因此,数值模拟已成为揭示物理机理并实现定量预测的重要途径。
但在深冷与高热流密度等极端工况下,现有数值模型与算法的适用范围
与不确定性边界仍难明确,尚不足以支撑工程所需的可靠预测与性能边
界评估。

多年来,基于欧拉网格的计算流体力学方法已在多相流动与相变传热领
域得到了广泛应用。
然而,面对流动、传热与相变的强非线性耦合与界面快速演化,这类方
法仍存在难以回避的共性瓶颈:
一方面,在界面剧烈运动与拓扑频繁变化的极端工况下,界面捕捉与重
构难以避免数值扩散与几何噪声,导致界面位置与曲率等几何信息计算
稳定性下降,引发界面通量误差累积并诱发非物理流动;
另一方面,在相变强源项作用下,质量与能量源项需在移动界面附近强
耦合施加,源项的局部施加与界面演化难以保持一致,易造成能量偏差
与热力学不一致;同时,强潜热交换与物性跨数量级跃变显著增强数值
刚性,使时间步长与收敛性受限并加剧能量闭合与稳定推进之间的矛盾。
上述问题在密度比悬殊或局部热流密度极高的条件下进一步凸显,且难
以通过加密网格或经验处理消除,导致计算成本显著上升、预测不确定
性难以约束,从而在一定程度上制约了传统网格法对非平衡相变耦合过
程的可靠预测与性能边界评估。

相较于欧拉网格方法,拉格朗日无网格的光滑粒子流体动力学(Smoothed 
Particle Hydrodynamics, SPH)\hl{无需界面重构,具备天然的界
面随动追踪能力,并对界面拓扑变化与多相大变形具有良好适应性},
成为模拟气液相变两相耦合过程的重要数值手段。
然而,面向深冷与高热流密度等强耦合工况,现有SPH方法仍存在关键
不足:
一方面,多数研究采用分块建模与分裂推进,将流动、传热与相变通过
状态量交换进行弱耦合,缺乏统一的离散框架与一致性构造,易导致动
量与能量更新不同步,难以在潜热强源项作用下保持能量闭合与热力学
一致性;
另一方面,在高密度比与界面邻域支撑退化条件下,界面耦合算子的稳
定性约束与相容性构造仍不充分,界面边界处理、通量交换与守恒更新
之间难以形成一致匹配,使数值稳定域显著收缩并易触发非物理界面扰
动。
上述瓶颈制约了SPH在强相变、高密度比等复杂条件下实现稳定、准确
且长期一致的数值演化,亟需面向强耦合过程完善其统一耦合框架与
界面稳定化方法。

因此,针对现有SPH框架在强耦合一致性、界面稳定性与高精度守恒离散
方面的关键瓶颈,有必要立足国家重大需求,结合申请人及所在团队的
前期相关工作积累,从\hl{统一粒子体系、统一算子构造与统一耦合
求解策略}出发,构建面向流动、传热与相变耦合过程的高精度强耦合
计算框架。
本项目以构建具备\hl{守恒一致性的多相统一SPH耦合方法}为目标,
重点解决潜热强源项作用下动量与能量一致耦
合更新、高密度比与界面邻域退化条件下的稳定界面传递与稳定域控制,
以及兼顾高阶精度与严格守恒的离散算子构建与稳定推进策略等关键问
题。
最终,项目将形成一套面向气液相变两相耦合问题的高精度多相统一SPH
方法与可扩展实现框架,为复杂工况下的定量预测与性能边界评估提供
高置信度工具,在理论与算法层面推进多物理场拉格朗日离散与强耦合
求解体系的发展。
该研究不仅具有重要\hl{基础科学意义},还将为航空航天热防护、深
冷燃料管理等国家重大工程需求提供高置信度的设计与优化工具,具备
显著\hl{工程应用价值}。

\section{国内外研究现状及分析}
气液相变两相流的数值研究以高保真计算为核心诉求:需在复杂界面演
化条件下实现动量与能量的守恒闭合,并在潜热交换过程中保持热力学
一致性。
围绕上述目标,国内外已形成多条数值路线,主要包括基于
\hl{欧拉网格}的界面捕捉追踪方法、\hl{网格-粒子耦合方法},以及
以SPH为代表的\hl{无网格粒子方法}。
不同路线在界面表示、相变闭合与耦合求解框架上侧重不同,适用工况
与数值稳定性表现亦不相同;
以下将从\hl{界面表示}、\hl{相变处理}与\hl{强耦合计算稳定性}三
个维度,对相关研究的典型做法与局限进行归纳比较,进一步凝练本项
目相关的关键科学问题与方法学缺口。

\subsection{界面表示与多相耦合建模方法}
气液相变两相流耦合计算的首要挑战在于\hl{稳定刻画相界面并在界面
处实现两相通量的一致交换}。
尤其在气相可压、液相近不可压且密度比与声速尺度差异显著的条件下,
界面处质量、动量与能量交换同时受到\hl{压缩性处理策略}、\hl{不
可压约束施加}与\hl{界面几何误差放大}的共同制约
\cite{nair2019simulations}。
在欧拉网格框架下,有限差分法(FDM)\cite{smith1985numerical}、
有限体积法(FVM)\cite{eymard2000finite}与有限元法(FEM)
\cite{reddy1993introduction}等方法在固定网格上离散两相控制
方程,并通常采用\hl{统一格式}与\hl{非统一格式}两类求解策略:
前者以同一套方程覆盖全域,通过弱可压近似兼容不可压相,但时间
步长受高声速相
限制而显著收缩\cite{hauke1994unified,billaud2011simple};
后者则分别求解不同压缩性相的控制方程,并通过耦合在界面满足
应力连续。因需在界面处协调不同离散与求解器,算法结构更为复杂
\cite{hauke1998comparative,nair2019simulations}。
无论采用哪类求解,网格法均需依赖界面捕捉与追踪算法刻画相界面,
如流体体积法(VOF)\cite{hirt1981volume,lafaurie1994modelling}、
水平集法\cite{osher1988fronts,sussman1994level}及相场方法
\cite{badalassi2003computation,chiu2011conservative,
yang2014phase}等,并已在气泡上升与液滴蒸发等典型气液两相问题
中得到广泛应用\cite{nikolopoulos2007numerical,
	strotos2008numerical,welch2000volume}。
然而当界面进入\hl{强拉伸、破碎/并合与多尺度细结构并存}的快速
拓扑演化阶段时,固定网格上的界面重构易引入数值扩散与几何噪声,
使法向与曲率等几何量难以稳定收敛,误差进而传递到界面力与界面
通量离散并诱发虚假流动;为抑制该类误差往往需要显著加密网格并
进一步收缩时间步长,计算代价陡增且在复杂拓扑变化下鲁棒性仍难
保证。	
因此,面向强变形、多拓扑且压缩性差异显著的气液两相耦合问题,
\hl{界面重构误差}与\hl{压缩性差异导致的时间尺度刚性}相互放大,
构成网格法难以回避的关键瓶颈。

为缓解固定网格在强变形问题中\hl{界面重构代价高}与\hl{鲁棒性
不足}的矛盾,国内外发展了网格-粒子耦合路线:在欧拉网格上求解
压力、温度等连续场以保持全局稳健性,利用拉格朗日粒子随动追踪
界面、刻画局部大变形与相间作用,并通过双向插值或外推实现网格
与粒子间的信息交换
\cite{liu2019conservative,li2020review,xu2021coupled,
	xu2023three,zhao2024coupling,xu2026numerical}。
然而,网格–粒子间的\hl{双向映射或投影}往往引入守恒误差并削弱
界面一致性:通量投影不匹配易导致界面漂移与误差累积,在高密度比、
强物性梯度乃至相变驱动界面快速迁移条件下,更易引发压力-热场
耦合的收敛与稳定性退化
\cite{li2020review,xu2026numerical}。
因此,该类耦合框架在极端工况下仍难同时兼顾投影守恒、通量一致
与稳定推进,其短板在界面快速演化与物性跃变并存时尤为突出
\cite{li2020review,xu2026numerical}。

与基于网格的方法相比,SPH作为拉格朗日粒子法可在无需额外界面
追踪的条件下随动刻画相界面,从而在自由表面、大变形等问题中展
现出天然优势,并已发展出多种多相SPH建模框架\cite{hu2006multi,
guo2024smoothed,pozorski2024smoothed,le2025smoothed}。
针对气液两相差异悬殊的密度、黏度与可压缩性,现有粒子法多相建
模大体形成两条主线:\hl{单流体同步求解}与\hl{分离式耦合求解}。
单流体同步求解在同一离散框架下同时推进两相场量\cite{hu2006multi,
guo2024smoothed,pozorski2024smoothed,le2025smoothed},
并通过密度求和与连续密度形式\cite{monaghan2005smoothed,
suresh2019comparative,park2020development}、界面物性平滑
修正\cite{hu2006multi}以及黎曼求解\cite{rezavand2020weakly,
	meng2020multiphase}等手段增强界面稳定性。
该路线实现简单、易扩展到复杂几何与自由表面,
但在\hl{强物性跃变与压缩性不一致}条件下依赖较强平滑与参数调节,
从而在\hl{界面清晰度、守恒一致性与时间步限制}间形成难以回避的
折中;当进一步引入界面热通量与相变源项时,数值扩散与时间尺度
刚性会更直接地表现为界面通量误差累积、能量闭合偏差或伪流增强。

分离式耦合求解针对不同相采用适配推进策略,如气相WCSPH与液相
ISPH/MPS,并在界面处通过\hl{压力与速度互馈边界条件}实现耦合
\cite{lind2015numerical,lind2016incompressible}。
该路线可在不平滑密度跃变的前提下保持真实物性比,但其稳定性高度
依赖\hl{界面条件的构造与离散相容性}:支撑域截断导致的界面邻域
退化、压力边界重构误差与表面张力离散会相互耦合,进而影响压力
解与通量交换闭合。
典型地,Duan等人\cite{duan2020incompressible}采用MPS-WCSPH
的不可压-可压耦合并以粒子注入刻画沸腾传质,实现了锐密度跃变与
气相体积膨胀的直接处理,但表明在考虑表面张力与强压梯度工况下,
界面混粒与稳定域收缩仍然是关键制约。
西安交大孙中国团队\cite{han2024incompressible}面向沸腾-冷凝
提出MPS-SPH框架并引入粒子分裂/合并以应对体积剧变与分辨率失配,
提升了可计算性,但粒子拓扑操作与界面邻域/通量交换的反馈会放大
耦合敏感性,仍需统一的守恒性与一致性约束。
近期,北京大学刘谋斌团队\cite{GUO2026114513}进一步从界面压力
与速度条件构造与PPE迭代效率入手提升可计算性,但在强物性差异与
界面快速演化下,界面边界重构噪声与计算代价之间的矛盾仍未根本消解。

总体来看,粒子法气液两相建模虽在界面随动方面具备优势,但在强
物性跃变与压缩性差异并存时,界面处的稳定耦合仍高度依赖压力/速度
边界条件的相容施加与通量交换的守恒闭合;一旦引入相变,界面质量
交换与局部分辨率变化还会进一步放大通量误差并压缩稳定域。
由此,\hl{如何在保持界面清晰与真实物性跃变的同时,实现界面通量
一致交换与守恒一致的稳定耦合推进},仍是粒子法气液两相建模需要
突破的关键堵点,并将直接决定后续相变传热模型与强耦合稳定性策略
能否在统一框架下可靠落地。
%
\subsection{相变传热建模与源项处理策略}
气液相变过程模拟的核心在于将界面处传热闭合为\hl{相间质量交换率},
并保证该质量源项与\hl{能量方程中的潜热项严格一致}。
在传统欧拉网格方法中,常在既有两相求解框架上引入蒸发/冷凝/沸腾
等闭合模型,将\hl{界面热通量}转换为\hl{相间质量通量},并以体积
分数方程、速度散度或能量方程源项等形式实现相变耦合
\cite{son1998numerical,welch2000volume,wang2015improved,
	safari2014consistent,nikolopoulos2007numerical,
	strotos2008numerical,tanguy2007level}。
例如,Samkhaniani在CF-VOF中结合Tanasawa传质模型模拟蒸汽泡冷
凝,指出预测结果对传质系数高度敏感
\cite{samkhaniani2016numerical},表明相变闭合在数值上对模型
参数选取与界面/热边界分辨率具有显著依赖性。
总体而言,网格法在小变形、界面相对平滑的相变问题中较为有效;
但当界面强变形并快速迁移时,\hl{界面定位与温度梯度误差会被源项
闭合直接放大},往往需要加密网格并收缩时间步来维持稳定,加剧稳定
性与计算代价的矛盾。

然而对于拉格朗日粒子法而言,\hl{满足相变闭合与潜热更新的守恒一致
性}是另一大挑战:
气液相变并非简单叠加一个能量源项即可获得稳健结果,而是会与两相耦
合推进在离散层面紧密缠绕——潜热吸放改变温度场与界面热通量,反过来
相变传质又会改写局部密度与体积分布,从而影响压力与速度耦合与两相
通量交换。
尤其在气液体系中,相变往往伴随显著的\hl{体积剧变},气相粒子快速稀
疏化,导致界面处邻域退化与多分辨率问题,使相变速率评估与潜热一致
闭合更易失稳。
围绕这一问题,SPH框架下发展出若干典型的相变传热建模与源项处理方法。

首先,通过范德华等热力学状态方程与内能演化将蒸发与沸腾等相变内
生为连续介质的相态转变,以避免显式构造界面相变通量。
Sigalotti等人在自适应SPH框架下模拟了液体蒸发与剧烈沸腾,并以
内聚项与毛细力产生表面张力\cite{sigalotti2014diffuse,
	sigalotti2015smoothed}。
此类方法形式统一,但通常呈现\hl{扩散界面},且对状态方程参数、分
辨率与时间步约束敏感;在强物性跃变与可压缩时间步限制并存时,界面
扩散与时间尺度刚性会共同掣肘其精度与鲁棒性\cite{duan2020incompressible}。

此外,直接将界面热通量闭合为质量交换率,并以粒子质量变化、相分
数演化或等效热容或焓法把潜热并入能量方程。
Yang和Kong提出允许粒子质量变化的SPH蒸发模型以表达界面传质,但
粒子尺度随之分化,需要粒子分裂/合并等拓扑操作以缓解分辨率失配
\cite{yang2017smoothed}。
相关研究也表明,当液相粒子逐渐减少或邻域退化时,温度梯度与热通量
评估精度会显著下降,进而反噬相变速率并引发误差累积\cite{xu2023numerical}。

同时,在保持锐界面与真实物性跃变的前提下同时处理相变与体积效应,
一些工作将界面质量通量落实为新相粒子注入、伪粒子累积或局部转相。
Das和Das通过引入伪粒子累积蒸汽质量并结合重分布来体现体积膨胀
\cite{das2015modeling};
Duan等人在不可压-可压分离式耦合框架中以气相粒子注入刻画沸腾传质,
并讨论了体积膨胀与多分辨率对稳定性的影响\cite{duan2020incompressible};
Han等进一步面向沸腾-冷凝建立MPS-SPH框架,引入粒子分裂/合并以
应对体积剧变与分辨率失配\cite{han2024incompressible}。
此类方法更贴近\hl{真实物性比}与\hl{锐界面},但其稳健性强依赖
相变粒子生成/转相区域的选择、界面邻域退化下的热通量评估,以及
拓扑操作对动量与能量守恒一致性的约束;处理不当易出现界面混粒、
局部误差累积与非物理振荡等问题
\cite{duan2020incompressible,han2024incompressible}。

综上,SPH在气液相变问题上的关键困难在于当\hl{强物性跃变}与\hl{体
积剧变}并存时,相变源项会同步改变密度/体积分布与粒子邻域拓扑,
使\hl{热通量评估、质量交换闭合与粒子拓扑操作}形成强反馈链;
任何经验系数、数值平滑或重分布误差都可能被放大为显著的相变
速率偏差与隐性守恒损失。
因此,\hl{如何在分辨率动态演化与邻域退化条件下,实现相变质量
源项与潜热项的严格一致闭合并保持守恒与热力学一致性},仍是现有
SPH气液相变建模中最具约束性的关键堵点。

\subsection{耦合求解与稳定时间推进策略}
由于压缩性差异,气液相的\hl{密度演化机理并不一致}:可压相密度
随压力与温度显著变化,而近不可压相还需满足\hl{零散度约束},
使界面处的边界交换与源项同步天然更敏感\cite{GUO2026114513}。
在网格法中,这类差异要么使统一格式的时间步长被声速差强烈限制,
要么使非统一格式必须在界面处协调不同离散与求解器,显著抬升耦
合复杂度\cite{GUO2026114513}。

SPH框架下面临同样问题:在弱可压缩SPH中,为维持弱可压缩假设常
需设置较大声速,从而使时间步长受声学尺度强约束\cite{fang2025accurate,
	GUO2026114513};
在不可压SPH中,压力泊松方程(PPE)的迭代收敛效率又直接决定
整体代价与可扩展性\cite{GUO2026114513}。
当进一步引入温度方程后,压力不仅受密度演化约束,还会通过\hl{状
态方程与温度场}发生反馈耦合;同时显式热扩散引入额外的稳定步长
限制,使多时间尺度矛盾更加突出\cite{fang2025accurate}。
上述时间刚性往往迫使算法采用分相/分步与多速率推进,但也会使
界面交换与源项同步更敏感,误差更易跨步传递与累积。

从稳定推进角度看,现有SPH相变实现多采用\hl{分步耦合的推进顺序}:
先进行两相动量/压力更新与界面条件交换,再更新温度并触发相变质量
源项,最后配合粒子操作修复体积变化导致的分布
退化\cite{han2024incompressible,lyu2023towards}。
在该过程中,一旦界面附近出现粒子稀疏、邻域不对称,或黏性/表面张
力等界面力使应力平衡对边界交换误差更敏感,便容易诱发压力噪声、
寄生流与界面混粒等失稳现象\cite{GUO2026114513}。
因此工程实现中常叠加多种稳定化手段,
例如背景压力\cite{guo2024smoothed,he2022stable}、
核梯度/一致性修正\cite{fan2025improved}以及粒子移位等分布
调控\cite{mokos2017multi},以缓解负压团聚、邻域退化与界面
噪声放大。

总体而言,强耦合气液相变问题的难点并不落在某一个子模型,而在于
\hl{时间步长刚性、状态方程--温度反馈、热通量评估、相变源项闭合
与粒子拓扑/分布操作}相互牵连;任一环节的显式分步、经验平滑或重
分布都可能在界面退化条件下触发链式放大,表现为相变速率漂移、
隐性守恒损失与寄生流增强。
因此,\textbf{\textcolor{B}{如何在强物性跃变与体积剧变并存时,
构建与时间推进一致的热--质闭合与稳定耦合框架,并在离散层面同时
约束守恒性与热力学一致性}},是极端工况可靠预测亟需突破的关键堵点。
%	
\section{主要问题及不足}
通过以上国内外研究现状分析,针对气液相变两相流耦合过程的数值
模拟,相关研究还存在以下几点不足:

\textbf{(1)\textcolor{B}{界面压力与速度传递离散不相容,
		导致耦合稳定性受限。}}
分离式两相SPH依赖界面边界条件的双向传递实现耦合,但在不可压-可
压分相推进下,界面压力条件构造与速度约束施加在离散层面难以保持
相容;叠加核支撑域截断、粒子扰动与邻域退化,界面离散误差被放大,
易诱发压力振荡、非物理流动与相间掺混等失稳现象。
现有方法多借助平滑或阻尼抑制振荡,但稳定化强度参数敏感,难以同
时兼顾稳定性与守恒一致性。
		
\textbf{(2)\textcolor{B}{相变闭合与潜热交换守恒不足,
		源项易引发耦合误差放大。}}		
相变建模中常将潜热以源项直接叠加,或在界面附近以插值或平滑更新,
使界面热通量与相变质量通量难以严格闭合,易出现温度过程、假相变
与能量漂移。
相变引起的质量转移与体积变化还会扰动压力求解与速度更新,放大耦
合误差并诱发相变速率的非物理波动,限制计算稳定域与适用范围。		
		
\textbf{(3)\textcolor{B}{粒子重构与多率推进缺少守恒约束,
		误差累积与稳定域受限。}}
气液相变会引起密度与体积分布快速变化,造成粒子稀疏/堆积与邻域
退化,常需引入分裂/合并等粒子重构以维持分辨率。
同时声学CFL与相变源项刚性等带来多时间尺度约束,统一时间步低效,
而多率推进又要求跨相、跨尺度同步更新与一致维护。
现有方法在变量重分配与跨尺度同步方面缺少守恒约束,易累积隐性质
量/能量误差并扰动界面耦合与相变更新,从而削弱稳定性。		
%	
\section{本项目拟开展的研究课题}
(1)探究界面压力条件与速度条件传递的离散相容与稳定耦合,面向
不可压-可压分相推进,分析界面边界重构、速度约束施加与两相推进
算子之间的匹配关系。

(2)研究热通量驱动的相变质量通量闭合与潜热交换守恒,面向界面
热通量、相变质量通量与能量更新的一致闭合关系,形成与压力求解
协同的相变更新策略。

(3)开展体积剧变条件下的守恒粒子分裂/合并与多时间尺度推进研究,
面向变量重分配与跨尺度同步更新,研究带守恒一致性约束的粒子重构
与多率推进耦合机制。
%
\begin{REF}
\subsection*{参考文献}
\vspace{-50pt}
\bibliography{ref}
\end{REF}

\newpage
