\section{立项背景及研究意义}
面向深冷推进剂管理、航天热防护与高热流密度散热等工况,沸腾冷却、
冷凝换热与液滴蒸发等气液相变过程广泛存在于推进剂储运及动力装备
预冷等关键环节\cite{kharangate2017review,
korsukova2025cfd,li2023recent,lee2012review,chen2024review}。
这类过程的共有特征是\hl{气液界面强变形与拓扑快速演化、物性随温
度跨数量级跃变以及潜热快速吸放强耦合},从而显著影响系统的热管理
效率、安全裕度与性能边界\cite{kharangate2017review}。
工程实践中仍大量依赖经验关联式与分段模型,但其适用范围有限、外
推不确定性难以约束,在新工况与新结构设计中或导致裕度过大或失效
风险上升\cite{mudawar2013recent,kharangate2017review}。
受空间分辨率、响应时间与安全及成本等因素限制,实验难以在大温差
与快速拓扑演化并存条件下实现多物理量同步全过程观测
\cite{li2023recent,korsukova2025cfd}。
因此,要支撑工程外推与性能边界评估,必须发展高保真数值模拟,其
关键在于\hl{界面附近强耦合过程的稳定求解}。

以欧拉网格为主的两相相变数值框架已在工程实践中得到广泛发展与应
用,推动了沸腾冷凝等过程的相关研究。
但在强耦合工况下仍面临两类结构性困难:一是界面快速拉伸、破碎与
并合使界面几何量与通量评估对数值扩散和几何噪声高度敏感,误差易
随时间推进累积并诱发非物理流动;
二是相变质量源项与潜热项需在移动界面附近一致闭合,叠加强物性跃
变带来的刚性,使时间步长与收敛性约束显著增强,加剧稳定推进与严
格守恒之间的矛盾
\cite{li2023recent,lee2012review,kharangate2017review}。
上述矛盾在强温差、高密度比与体积剧变等场景中更为突出,单纯依赖
网格加密与经验修补往往导致代价攀升而不确定性边界仍难清晰给出。

以SPH为代表的拉格朗日无网格粒子方法因界面随动、对拓扑变化与大
变形的天然适应性,被认为是多相强变形等问题的重要路线
\cite{sakai2020recent,bagheri2024review}。
更重要的是,在同一粒子体系表示下同时承载界面运动、传热与质量交
换,使得在离散层面对界面通量一致交换、热-质闭合守恒与跨尺度
同步推进施加一致性约束具有更直接的结构条件
\cite{le2025smoothed,bagheri2024review}。
但在强温差、高密度比与体积剧变并存的气液相变中,现有粒子法相变
模拟仍普遍受限于:界面邻域退化下热通量与界面通量评估误差放大,
相变传质与潜热更新难以严格一致闭合,以及多时间尺度下跨相同步推
进易触发压力噪声、界面混粒与守恒漂移等失稳
\cite{sakai2020recent,bagheri2024review,le2025smoothed}。
这表明粒子法的潜力能否转化为工程相关区间的可靠预测,关键仍取决
于建立强耦合条件下通量交换、闭合守恒与耦合推进之间可验证的一致
性构造。

因此,本项目聚焦上述\hl{算法层面}的关键堵点:面向\hl{强温差}、
\hl{高密度比}与\hl{体积剧变}并存的气液相变强耦合过程,
以\hl{可控误差}与\hl{守恒一致}为目标,在SPH框架内建立界面通
量一致交换、热-质闭合守恒与稳定时间推进的离散约束,形成相应的
强耦合求解流程并开展系统验证。
本项目不聚焦成核机理刻画与全工况工程外推,预期为深冷燃料管理
与航天热防护等场景提供更可信的数值评估工具,并积累多物理场拉
格朗日离散的守恒一致与热力学一致方法学。

\section{国内外研究现状及分析}
气液相变两相流的数值模拟,核心目标是在极端工况下对\hl{界面演
化}、\hl{动量传递}与\hl{能量交换}给出可验证的高保真预测。
围绕这一目标,国内外主要形成三条路线:基于\hl{欧拉网格}的界面
捕捉与追踪方法、\hl{网格-粒子耦合}方法,以及以SPH为代表
的\hl{无网格粒子方法}。
不同路线在界面表示与相变闭合上各有优势,但在强耦合条件下往往受
到离散结构带来的系统性约束。
以下从\hl{界面表示}、\hl{相变处理}与\hl{强耦合稳定推进}三个维
度归纳典型做法与局限,据此凝练本项目相关的科学问题与方法学缺口。

\subsection{界面表示与多相耦合建模方法}
气液相变两相流耦合计算的首要挑战在于\hl{在界面剧烈演化时,仍能
稳定评估界面几何量并实现界面处通量的一致交换}。
在欧拉网格框架下,流体体积法(VOF)\cite{hirt1981volume,
lafaurie1994modelling}、水平集\cite{osher1988fronts,
sussman1994level}与相场方法\cite{badalassi2003computation,
chiu2011conservative,yang2014phase}等已广泛用于气泡上升、液
滴蒸发等多相问题\cite{nikolopoulos2007numerical,
	strotos2008numerical,welch2000volume}。
这类方法在界面平滑与拓扑变化有限的条件下已形成较成熟的数值体系;
但在界面强拉伸、破碎与并合且伴随多尺度细结构时,界面捕捉与重构
易产生数值扩散与几何噪声,其误差会污染到法向与曲率等界面几何量
计算,进而放大界面通量的离散误差,表现为非物理流动与并收缩可稳
定计算参数范围\cite{korsukova2025cfd,li2023recent}。
为维持稳定性,需显著加密网格并收缩时间步,计算代价陡增且工程可
用性降低。
为缓解网格界面捕捉与重构在快速迁移过程中的误差累积,网格与粒子
耦合方法通常在欧拉网格上求解压力、温度等连续场,并用粒子随动追
踪界面,通过插值与投影实现信息交换\cite{liu2019conservative,
li2020review,xu2021coupled,xu2023three,zhao2024coupling,
xu2026numerical}。
该类方法在一定程度上可缓解界面数值扩散,但跨网格与粒子的映射难
以同时满足守恒性与界面一致性;
在高密度比、强物性跃变及相变驱动界面快速迁移时,界面动量通量与
热通量误差更易累积,并集中表现为压力场与温度场耦合求解的收敛性
退化\cite{li2020review,xu2026numerical}。
因此,在极端工况下,如何以可控误差实现界面通量交换,仍是网格及
其耦合框架的关键瓶颈。

相比之下,SPH作为拉格朗日粒子法可随动刻画相界面,对自由表面与
大变形具有天然适应性,并已发展出多种多相建模框架
\cite{hu2006multi,lind2015numerical,lind2016incompressible}。
针对气液两相在密度、黏度与可压缩性上的显著差异,多相SPH主要形
成两类策略。	
其一是\hl{单流体同步求解}:在统一离散框架内推进两相场量
\cite{hu2006multi,pozorski2024smoothed,le2025smoothed},
并通过密度求和或连续密度形式\cite{monaghan2005smoothed,
suresh2019comparative,park2020development}、界面物性平滑
\cite{hu2006multi}及黎曼求解\cite{rezavand2020weakly,
meng2020multiphase}等增强界面稳定性。
该路线实现相对简洁,但在强物性跃变与压缩性差异并存时往往需要引
入更强的平滑与参数调节,从而在界面清晰度、守恒一致性与时间步限
制之间形成难以避免的折中;
引入相变后,界面热通量与质量通量的误差也更容易在时间推进中被放
大。

其二是\hl{分离式耦合求解}:针对不同相采用更适配的推进策略,例
如气相WCSPH与液相ISPH或MPS,并在界面处通过压力与速度条件互馈实
现耦合\cite{lind2015numerical,lind2016incompressible,
duan2020incompressible,han2024incompressible,guo2024smoothed}。
该路线可在不平滑密度跃变的前提下保留真实物性比,但稳定性高度依
赖界面条件的构造及离散相容性:支撑域截断与粒子分布不均会导致界
面邻域退化,压力边界重构与界面力计算误差放大,易诱发压力噪声、
界面混粒并推高耦合代价。
典型工作表明:在强表面张力与大压梯度条件下,界面混粒与稳定域收
缩仍较难避免\cite{duan2020incompressible};
同时,粒子拓扑操作与界面通量交换之间可能形成反馈,进一步放大耦
合敏感性\cite{han2024incompressible}。
改进界面条件构造并提高压力泊松方程(PPE)求解效率可提升可计算
性\cite{GUO2026114513},但在强物性差异与快速界面演化下,边界
重构噪声与计算代价之间的权衡仍然突出。

总体而言,粒子法在界面随动方面具备优势,但在强物性跃变与压缩性
差异并存的两相耦合条件下,其可靠性关键在于\hl{界面条件、通量交
换与两相推进算子在离散层面的相容构造}。该问题在分离式耦合框架中
尤为突出。
%
\subsection{相变传热建模与源项处理策略}
气液相变过程模拟的关键在于将\hl{界面热通量闭合为相间质量交换率,
并与潜热项保持离散一致}。在欧拉网格方法中,通常在两相求解框架上
引入蒸发、冷凝或沸腾等闭合模型,将界面热通量转换为相间质量通量,
并以体积分数方程、速度散度或能量方程源项等形式实现相变耦合
\cite{son1998numerical,welch2000volume,wang2015improved,
	safari2014consistent,tanguy2007level}。
这类框架便于与现有求解器集成,但相变预测对界面定位、温度梯度与
模型参数较为敏感\cite{samkhaniani2016numerical}。
在界面强变形与快速迁移时,界面定位误差与梯度误差会直接传递到热
通量评估,进而放大质量源项与潜热项的不一致,常需加密网格并收缩
时间步以维持稳定,导致计算代价上升且适用工况受限。
网格-粒子耦合方法可借助粒子随动追踪界面或相变区域以减弱界面数值
扩散,但源项与通量在网格与粒子之间的映射难以同时满足守恒性与界
面一致性,在强迁移与强跃变下闭合误差更易累积
\cite{li2020review,xu2026numerical}。

对拉格朗日粒子法,SPH在传热离散方面已形成面向复杂边界与多材料
界面的基本工具,可在粒子体系内离散热传导并通过界面修正保证异质
材料间热流连续\cite{cleary1999conduction},结合一致性修正与
边界处理可改进温度梯度与界面热通量评估
\cite{chen1999corrective,jeong2003smoothed},隐式推进有助
于提升长时传热计算的稳定性\cite{rook2007modeling,
chaniotis2002remeshed,yang2019numerical,szewc2011modeling}。
然而在气液多相体系中,物性跨量级跃变与温度场不连续使界面附近热
通量评估误差显著放大,并通过与压力、速度更新的耦合反馈影响相变
闭合的稳定性。
近期多相热力学黎曼SPH在处理不连续温度场与热–流体耦合方面展示出
精度潜力\cite{fang2025accurate},但其与相变质量闭合及分相推
进之间的离散仍缺乏一致性约束。

面向气液相变,现有SPH建模大体可分为三类。
第一类以状态方程与内能演化将相变内生为连续相态转变,从而避免显
式构造界面相变通量\cite{sigalotti2014diffuse,
sigalotti2015smoothed};形式统一,但往往呈扩散界面,且对参数、
分辨率与时间步较为敏感。
第二类将界面热通量闭合为质量交换率,并在既有粒子上通过质量变化
或相分数连续更新物质与能量\cite{yang2017smoothed};
实现直接,但在邻域退化或粒子尺度分化时,温度梯度与热通量误差易
被放大,进而导致相变速率偏差与误差累积\cite{xu2023numerical}。
第三类在粒子层面显式承载相间质量交换,如粒子注入、伪粒子累积或
局部转相,并常配合分裂与合并处理体积变化\cite{das2015modeling,
duan2020incompressible,han2024incompressible};
更贴近锐界面与体积效应,但对相变区域判定、退化邻域下的热通量评估
以及拓扑操作的动量与能量守恒约束高度敏感,处理不当易诱发界面混粒
与非物理振荡\cite{duan2020incompressible,han2024incompressible}。

因此,在强物性跃变与体积变化并存时,如何在邻域退化与分辨率演化
条件下,实现\hl{相变质量通量与潜热交换的严格一致闭合},并与守
恒更新和推进稳定性相协调是现有SPH相变建模的关键缺口。

\subsection{耦合求解与稳定时间推进策略}
气液相变强耦合稳定求解的难点在于\hl{分步推进的串联使跨步不同步
误差在界面附近被放大并累积}。
工程实现中常将动量与压力更新、界面条件交换、温度演化、相变源项
更新,以及为维持粒子分布而引入的粒子操作按顺序拼接推进
\cite{han2024incompressible,lyu2023towards}。
各子步单独可控,但在界面邻域退化与强源项并存时,界面处的同步误
差会跨步传递并累积,最终表现为压力噪声、非物理流动、界面混粒以
及隐性守恒漂移等失稳\cite{GUO2026114513}。


上述跨步误差累积的根源之一,是两相压缩性差异带来的时间尺度不一
致。
气相密度随压力与温度显著变化,而近不可压液相还需满足零散度约束,
导致界面处的边界交换与源项同步更为敏感\cite{GUO2026114513}。
在网格框架中,常需通过更严格的时间步限制或分裂与多速率策略来维
持稳定,并在界面协调不同求解器以保证同步与守恒,从而显著增加实
现复杂度与计算代价\cite{GUO2026114513}。
类似地,网格-粒子耦合方法虽可引入随动界面刻画,但跨表示的同步更
新与映射误差在多时间尺度下同样会放大耦合敏感性
\cite{li2020review,xu2026numerical}。

在SPH框架下,时间推进矛盾表现得更为直接。
弱可压缩方法为维持弱可压假设通常需要设置较大声速,时间步长受声
学尺度强约束\cite{fang2025accurate,GUO2026114513};
不可压方法依赖压力泊松方程迭代,其收敛效率决定总体代价与可扩展
性\cite{GUO2026114513}。
当引入温度方程后,状态方程与温度反馈会进一步耦合压力与能量更新,
显式热扩散也带来额外稳定步长限制\cite{fang2025accurate},使
多时间尺度矛盾进一步地突出。
为提升可计算性,已有工作常叠加背景压力
\cite{guo2024smoothed,he2022stable}、
一致性修正\cite{fan2025improved}
与粒子移位\cite{mokos2017multi}等稳定化手段,但多以额外耗散
或重排换取稳定,参数依赖较强,且与严格守恒的一致性存在张力。

综上,在界面邻域退化与强源项并存的极端条件下,\hl{如何使界面条
件交换、热-质源项更新与两相推进在同一时间推进框架内保持同步一致,
并将守恒漂移约束在可控范围内},仍是强耦合气液相变稳健预测的关键
堵点。

\section{主要问题及不足}

通过以上国内外研究现状分析,针对气液相变两相流耦合过程的数值
模拟,相关研究还存在以下几点不足:

\textbf{(1)\textcolor{B}{界面压力与速度传递离散不相容,
导致耦合稳定性受限。}}分离式两相SPH依赖界面边界条件双向传递,
但在不可压-可压分相推进下,界面压力边界重构、速度约束施加与两
相推进算子缺少统一的离散匹配;
叠加核支撑域截断、粒子扰动与邻域退化,界面条件误差易放大并传递
到通量交换,诱发压力振荡、非物理流动与相间掺混。
现有方法多采用平滑或阻尼抑制振荡,但稳定化强度参数敏感,难以同
时兼顾稳定性与守恒一致性。

\textbf{(2)\textcolor{B}{相变闭合与潜热交换守恒不足,源项
易引发耦合误差放大。}}相变建模常将潜热源项直接叠加,或在界面
附近以插值或平滑更新,使界面热通量与相变质量通量难以严格闭合,
易出现温度过冲、假相变与能量漂移。
质量转移与体积分布变化又会改变局部密度并影响压力与速度更新,
与界面条件交换形成反馈,导致相变速率非物理波动并收缩稳定域。
现有做法多通过加厚相变带或平滑热通量提高稳健性,但会引入额外扩
散或隐性守恒误差。

\textbf{(3)\textcolor{B}{粒子重构与多率推进缺少守恒约束,
误差累积与稳定域受限。}}气液相变引起密度与体积分布快速变化,
易导致粒子稀疏与邻域退化,往往需要分裂与合并等重构维持分辨率;
同时声学CFL与源项刚性等带来多时间尺度约束,统一时间步低效,
而多率推进要求跨相同步更新与一致性维护。
现有方法在重构引起的变量重分配与跨尺度同步更新中缺少守恒一致
约束,隐性质量与能量误差易累积并扰动界面耦合与相变更新,削弱
长期稳定性。

\section{本项目拟开展的研究课题}
(1)研究界面压力边界重构与速度约束施加的离散相容与稳定耦合,
面向分相推进,建立界面条件交换与两相推进算子之间的匹配关系。

(2)明确热通量、相变质量通量与能量更新的严格一致性关系,
并给出与压力/速度更新相协调的相变耦合方式与更新顺序。

(3)研究体积剧变条件下粒子分裂/合并与多时间尺度推进的守恒约束,
面向变量重分配与跨尺度同步更新,构造带守恒一致性约束的粒子重构
与多率推进耦合机制。
\begin{REF}
\subsection*{参考文献}
\vspace{-50pt}
\bibliography{ref}
\end{REF}

\newpage
