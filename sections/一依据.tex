\section{研究背景及科学意义}
%
流-热-相变耦合过程广泛存在于深冷换热、蒸发冷却、薄膜沸腾、喷雾冷却、
液滴蒸发等典型界面传热现象中,是影响多相流动与能量传输的核心科学问题。
耦合过程中,界面常发生剧烈拓扑演化,密度、粘度等物性参数可出现数个
量级的突变;同时,相变潜热的快速释放或吸收进一步诱导强烈的温度梯度
和显著能量输运,并驱动非平衡条件下复杂的界面动力学行为。
这类多场相互作用现象在航空航天热防护、深冷燃料管理、高效冷却系统及
新型能源装置中具有关键影响,其耦合物理规律直接决定运行安全性、热管
理效率及性能极限。
\textbf{\textcolor{B}{因此,深入揭示流-热-相变耦合机制,是推动
		多相传热理论发展与理解复杂工况下物理规律的关键基础。}}
受制于成本、安全性及测量分辨率等因素,传统实验观测难以全面覆盖上述
复杂耦合过程的多尺度演化行为,因此数值模拟成为揭示流–热–相变耦合
机理并实现其演化过程定量预测的重要研究手段。

\textbf{\textcolor{R}{这里要加一张实际应用场景的拼图来阐述真实
		的工程应用背景及物理意义。}}

多年来,基于欧拉网格的计算流体力学方法,如体积函数法、相场法与水
平集法等,在多相流动与相变传热模拟中得到了广泛应用。
然而,在流-热-相变强耦合条件下,这类方法仍面临一系列难以回避的固
有瓶颈。
在剧烈界面运动、拓扑频繁变化或相变引起体积分数突变等极端工况下,
界面重构算法难以同时保证几何保真度与曲率计算精度,界面质量守恒性
受到破坏,易诱发非物理的虚假流动。
同时,移动界面附近潜热源项的离散处理通常依赖经验插值或弥散化模型,
容易引入显著的非物理能量误差,导致局部热力学不一致甚至数值不收敛。
此外,在热传导与流动之间存在强非线性交互的情况下,显式或半隐式耦
合策略对时间步长极为敏感,在多时间尺度并存时易产生严重的数值刚性
与稳定性瓶颈。
特别是在密度比悬殊或局部热流密度极高的严苛条件下,上述问题进一步
放大,在一定程度上制约了传统欧拉网格方法在复杂界面与非平衡过程中
的建模精度与预测置信度。
为弥补欧拉网格方法在复杂界面演化中的不足,已有研究尝试引入网格–粒
子耦合策略,但在强相变与多尺度耦合条件下,其一致性构造与数值稳定
性仍面临显著挑战。

相较于欧拉网格方法,拉格朗日无网格的光滑粒子流体动力学法(Smoothed 
Particle Hydrodynamics, SPH)\textbf{\textcolor{B}{因无需界面
		重构,具备天然的界面随动追踪能力,并对界面拓扑变化与多相大变形具
		有良好适应性,被认为是模拟流-热-相变耦合过程的潜在重要数值手段。}}
然而,现有SPH方法在处理多物理场强耦合问题时仍存在显著不足。
首先,多数研究仍将流动、传热与相变过程解耦建模,物理模块之间主要
通过状态量交换实现弱耦合,缺乏统一的离散框架与一致性构造,易导致
动量与能量演化不同步,难以满足严格的热力学一致性要求。
其次,不同物理模块往往采用不统一的核函数形式与数值稳定化策略,在
多相界面处易诱发压力的非物理振荡,并在高密度比条件下引发严重的数
值不稳定性。
上述问题使得现有SPH方法难以在强相变、高密度比等复杂耦合工况下实现
稳定、准确且长期一致的数值演化,制约了其由潜力方法向工程实用工具
的进一步发展。

因此,针对传统网格法和现有SPH框架在耦合一致性、数值稳定性与计算
精度上的固有瓶颈,从\textbf{\textcolor{B}{统一粒子体系、统一算子
		构造与统一耦合求解策略}}的角度出发,构建一套能够实现流动、传热与
相变过程高精度、强耦合演化的计算框架,已成为亟待解决的关键问题。
本项目以构建\textbf{\textcolor{B}{具备热力学一致性的多相统一SPH
		框架}}为目标,拟重点解决潜热源项的动量–能量耦合一致性、高密度比
条件下的界面稳定性以及高阶精度守恒算子的构建等关键挑战。
最终,项目将形成一套面向流-热-相变耦合问题的高精度多相统一SPH方法,
在理论层面与底层算法层面推动多物理场拉格朗日离散理论的发展,构建
统一、一致且具备良好可扩展性的数值计算框架。
该研究不仅具有重要的\textbf{\textcolor{B}{基础科学意义,还将为
		航空航天热防护系统、深冷燃料管理等国家重大工程需求提供高置信度的
		设计与优化工具,具备显著的工程应用价值}}。
%
\section{国内外研究现状及分析}
针对\textbf{\textcolor{R}{沸腾两相流的}}流–热–相变耦合过程涉及
界面剧烈拓扑演化、物性参数的跨尺度突变以及
强非线性传热等复杂物理特征,且常发生于高温、深冷等极端工况条件下,
使得界面位置、相变速率与温度场分布呈现显著的时空非均匀性。
\textbf{\textcolor{B}{在此背景下,能够同时描述复杂界面演化、严格
满足动量与能量守恒,并保持潜热交换热力学一致性的高保真数值模拟方
法,已成为研究流–热–相变耦合机理及其演化规律的关键技术途径。}}
然而,针对高度耦合的流–热–相变问题,不同数值方法在界面表示方式、
潜热处理策略及耦合求解框架等方面存在显著差异,其适用工况范围与数
值稳定性表现亦不尽相同。
综合国内外相关研究进展,围绕流–热–相变耦合问题的数值求解策略主要
形成了三类技术路径:\textbf{\textcolor{B}{基于欧拉网格的数值方法、
融合欧拉与拉格朗日优势的网格–粒子耦合方法,以及以SPH为代表的无网
格粒子方法。}}
以下将围绕上述三类方法的发展脉络与核心机制,系统梳理其在处理流–热–相
变等多物理场强耦合过程中的典型做法与代表性成果,重点对比分析其在界面
描述能力、守恒一致性与数值稳定性等方面的共性瓶颈,进而凝练亟需突破的
关键科学问题,为后续构建高精度多相统一计算框架明确技术路线与突破口。

\textbf{\textcolor{R}{放在为什么要做数值模拟那里:由于其复杂性和
困难性,气液两相流的实验研究在可视化、测量、规模、可重复性和代表性
方面都受到限制。
同时,成本和时间投入也是限制实验研究的主要因素。
因此,研究人员采用数值模拟方法研究气液两相流,并提出了气液两相流数值
方法的研究方向。}}

\textcolor{B}{多相流是复杂的物理现象~\cite{brennen2005fundamentals},
涉及界面处动量和能量的耦合、相变、表面张力以及其他复杂现象。
在连续介质力学层面,多相流系统详细描述了不混溶相之间的界面动力学、
	破碎或聚结引起的拓扑变化以及润湿效应引起的界面-壁相互作用。
	多相流模拟在科学和工程领域应用广泛,并在各种应用中发挥着重要作用
	~\cite{sattari2020multiphase}。
	由于不同流体相之间存在不连续性以及它们压缩性各异,因此,在数值方法
	中精确模拟多相流面临着诸多挑战~
	\cite{rezavand2020weakly,lyu2021study}。
	尤其对于具有较大密度比的可压缩/不可压缩耦合多相流的模拟而言,
	这给数值模拟带来了三个主要挑战。
	首先,密度在不可压缩相中是恒定的,而在可压缩相中,密度会随压力和
	温度而变化。
	其次,流体速度零散度条件必须应用于不可压缩相,这要求精确的界面跟
	踪,并在界面附近模拟此约束时提出了数值挑战。
	最后,可压缩相中压力波的传播会使多相流界面处的正应力平衡变得复杂,
	并常常导致模拟失败。因此,用于模拟可压缩-不可压缩两相流问题的数值
	方法相当有限。}	
%
\subsection{基于网格法的多相相变模拟方法及其耦合局限}
传统的基于网格的 CFD 方法试图使用统一或非统一的方法来解决这些问题。
统一方法是指对不可压缩相(水)和可压缩相(空气)使用同一组控制方程。
\textcolor{B}{不可压缩相假设弱可压缩性,因此在整个区域内求解可压缩
	纳维-斯托克斯方程。这类方法将不可压缩性视为一种极限情况。
	然而,由于各相之间的压缩性不一致以及声速差异较大,时间积分中使用的
	时间步长被限制在一个非常小的值~\cite{hauke1994unified,hauke1998comparative}。
	使用统一方法模拟具有不同压缩性的流体时,会面临时间
	步长过小的问题。
	另一方面,非统一格式针对具有不同压缩性的流体求解不同的控制方程,
	并采用显式或隐式耦合来实现应力平衡。不同的控制方程组需要不同的
	数值格式,这使得求解过程更加复杂。}
基于欧拉网格的数值方法是研究流-热-相变问题中应用最为广泛的技术路
线之一,在过去几十年中得到了广泛发展与应用。
网格法中各个网格点通过拓扑映射以预定义的方式连接在一起,主要包括
有限差分法(FDM)~\cite{smith1985numerical}、
有限体积法(FVM)~\cite{eymard2000finite}
和有限元法(FEM)~\cite{reddy1993introduction}。
其基本思想都是在固定计算网格上求解多相流动与能量输运控制方程,并
通过界面捕捉或追踪策略描述界面演化过程,从而实现流动、热传导与相
变过程的耦合模拟。
早期的研究主要依赖流体体积法(Volume of Fluid, VOF)进行界面捕捉 
\cite{hirt1981volume,lafaurie1994modelling},通过计算单元内各相
的体积分数来追踪界面位置,尤其在气泡流等两相流中得到了广泛的使用。
Nikolopoulos等人~\cite{nikolopoulos2007numerical}~采用有限体积
法结合流体体积法研究了正庚烷和水滴撞击热表面的蒸发过程。
Strotos等人~\cite{strotos2008numerical}~采用流体体积法研究了水
滴在低韦伯数下沉积在加热表面上的蒸发过程。
Welch和Wilson~\cite{welch2000volume}~则基于流体体积法推导了膜沸
腾模型。
该方法在大尺度界面运动问题中具有较好的守恒性,但在处理剧烈界面拓扑
变化及曲率计算方面存在一定的局限性。
其次水平集法(Level Set)~\cite{osher1988fronts, sussman1994level}
通过定义隐式函数表示界面,实现了拓扑变化的自然处理,并能够直接计算
界面法向量与曲率,也为气液相变及两相流问题提供了良好的界面动力学描述。
Tanguy等人~\cite{tanguy2007level}~结合了水平集方法和虚拟流体方法,
以捕捉界面运动并处理界面处的条件。
Son和Dhir~\cite{son1998numerical}~基于水平集方法建立了膜沸腾模型。
其次,相场方法(Phase Field)~\cite{badalassi2003computation,
	chiu2011conservative,yang2014phase}~通过连续变量描述界面区域,将
界面动力学与热传导的耦合纳入统一方程框架,在熔化/凝固、喷雾冷却以
及薄膜沸腾等问题中得到了广泛应用。
此外,结合格子玻尔兹曼方法(Lattice Boltzmann method, LBM)
\cite{safari2014consistent,wang2015improved},
前沿跟踪方法(Front Tracking)~\cite{tryggvason2001front}~等
的多种网格方法也被广泛用于多相流及沸腾相变的数值模拟中。
Juric和Tryggvason~\cite{juric1998computations}~基于前沿追踪
方法提出了汽液相变模型。
在这些网格划分方法中,沸腾的传质速率被转化为速度散度的源项。
换句话说,沸腾会导致界面两侧出现速度差;因此,体积膨胀可以通过
欧拉网格上的界面运动直接考虑。
在数值方法中,欧拉-欧拉方法[X]和欧拉-拉格朗日方法[X]被广泛应用
于两相流问题,因为在这些问题中,相界面并非主要关注点。
在精确表征相界面方面,流体体积法(VOF)、水平集法(LS)和GFM-流
体法(GFM)等方法具有明显的优势。
\textbf{\textcolor{B}{然而,VOF和LS方法容易受到界面扩散的影响,
		导致界面表征的清晰度和精度下降。这些方法对表征复杂界面的网格尺寸
		和保证结果准确性的参数要求非常高。}}

\textbf{\textcolor{R}{这里要加一下网格法耦合特殊处理方法的图,
		其次也要加一些结果来说明这个方法在计算精度和效率方面是存在问题的。}}

尽管网格法由于数值稳定性较高、边界条件处理成熟且对大尺度连续
问题能够较为有效求解而在多物理场耦合模拟中具有广泛的应用,且对于
涉及小变形相变问题时,能够提供相对精确的结果。
在深冷燃料管理中,液氢及液氧在储存与输运过程中容易发生过冷沸腾、
闪蒸及气液相分离,网格法通过捕捉气液界面及温度场分布,为预测储箱
压力及燃料输运提供了重要工具。
在航空航天热防护领域,烧蚀防热层及相变冷却材料中涉及材料熔融、气
化及多孔介质内流动的耦合过程,网格法能够模拟流-热-相变耦合对热防
护性能及安全裕度的影响。
但它们需要专门的处理方法来精确追踪和捕捉界面,导致计算成本高昂。
在处理自由界面大变形、高密度比及强耦合沸腾相变问题时仍存在显著瓶颈。
界面拓扑变化剧烈时,传统重构算法难以保证几何保真度及曲率精度,易
导致非物理流动或界面质量丢失,界面拓扑由于破碎合并时易出现数值扩
散,导致界面拓扑变化不够鲁棒。
潜热源项的离散通常依赖经验插值或弥散化处理,移动界面附近可能出现
局部能量不守恒或热力学不一致,影响数值收敛性,界面追踪与热力学一
致性处理容易产生假相变、温度过冲等。
此外,在处理流动、热传导与相变强耦合时,显式或半隐式策略在多时间
尺度下对时间步长高度敏感,容易出现数值刚性或不稳定现象,尤其在
密度比悬殊或局部热流密度极高的条件下。物性跨尺度变化导致稳定性受
限,在深冷、高温都难以处理。
\textbf{\textcolor{B}{因此,尽管网格法在工程应用中仍然是重要手
		段,其在高精度、多场强耦合问题下的适用性和可靠性仍面临方法本身难
		以克服的瓶颈问题。}}

\textcolor{B}{由于传热的物理问题通常涉及复杂的流体流动,特别是
	涉及多种材料和多个界面时,这些流动过程会导致界面飞溅和破裂,而
	热过程和化学过程又会引入进一步的复杂性,所有这些有时都会给基于
	网格的方法带来挑战,因为难以准确地捕捉界面。}
%
\subsection{欧拉-拉格朗日耦合方法及其一致性挑战}
%
网格-无网格耦合方法是在传统欧拉网格方法与拉格朗日无网格方法各具
优势而又存在显著局限的背景下发展起来的一类重要计算策略。
其核心思想是在欧拉网格上求解连续场,如流场、温度场、电磁场或
压力场等,同时使用拉格朗日例子方法,如SPH,MPS,MPM等方法,描述
界面运动、大变形结构或局部相变质量交换,从而在保持欧拉框架计算
稳定性的前提下,实现对强非线性拓扑变化、多场耦合及大位移现象的
更高保真模拟,网格负责大域求解,粒子负责局部细节。
最早的网格-无网格耦合尝试在流固耦合领域,欧拉网格用于求解不可
压流体,固体结构采用拉格朗日有限元表示,例如FEM-SPH耦合方法。
随着研究对象从单一结构大变形拓展到多物理场耦合,网格-无网格
方法开始被系统引入沸腾相变多相流等复杂问题中。
基于网格的方法能够很好地预测流场变量的分布,而基于粒子的方法
在相界面追踪方面具有天然优势。
近年来,许多研究将这两种方法结合起来,开发了一系列网格-粒子耦合
模拟方法~\cite{liu2019conservative,li2020review,xu2023three,
	zhao2024coupling,xu2026numerical},并且用于气泡等沸腾两相流等
模拟。
Xu等人~\cite{xu2021coupled,xu2023three}~开发了一种ISPH-FVM耦
合方法并将该耦合方法从二维扩展到三维,并成功应用于静态流体中气
泡上升和合并的模拟分析。
Zhao等人~\cite{zhao2024coupling}~开发了一种SPH-FVM耦合方法来
模拟电场对液滴形状的影响。在该耦合方法中,两相流的控制方程采用
SPH方法求解,电场则采用FVM方法求解。
这些耦合方法在不牺牲数值稳定性的前提下可以处理界面断裂、重接触、
微尺度蒸发、多相破碎与飞溅等网格法难以处理的现象。
在喷雾蒸发、液滴撞壁、薄膜沸腾等现象中都得到了一定的应用。
\textcolor{B}{SUN:研究人员还探索了将粒子方法与其他无网格技术
	相结合来模拟气液相变的方法,例如使用流动方向局部网格的无网格平流
	(MPS-MAFL)和状态方程方法。
	Hochstetter和Kolb~\cite{hochstetter2017evaporation}~将粗网格
	用于气相模拟,将基于SPH的液相模拟耦合,以模拟蒸发和冷凝过程。
	Tian等人分别使用MPS和状态方程描述液相和气相,模拟了不同过
	冷度下的单气泡冷凝过程。}

\textbf{\textcolor{R}{这里要加一下两种方法耦合处理的图,其次也
		要加一些结果来说明这个方法在计算精度和效率方面是存在问题的。}}

尽管这些粒子与网格的耦合方法经过一定的发展,在模拟多物理场中的
复杂相变流动方面具有巨大的潜力,然而若干核心挑战仍普遍存在并限
制其在强耦合相变问题中的可靠性与精度。
网格与粒子之间的双向投影往往是精度与守恒性的薄弱环节,但高精度
投影算子与修正项的引入将大幅提升复杂度与计算成本。
粒子表示的统计噪声与映射带来的高频误差会在热场计算中被放大,导
致温度场震荡或潜热耦合不稳定。
混合方法在高密度比或高物性梯度,如深冷液体与气相条件下的稳态性
尚待明晰,引入的界面跳跃和物性不连续性会影响压力与热场求解的
收敛性。
同时相变界面位置在网格和粒子之间难以保证完全一致,导致界面
产生漂移,网格与粒子之间的物性场离散不一致会导致温度场不连续。

总体来看,网格-无网格耦合方法已成为连接网格法与纯粒子法的关键
桥梁。它们把粒子的局部灵活性与网格的全局守恒优势结合起来,在
处理界面剧变与多场耦合方面展示出明显潜力。
但是由于投影守恒性、热质量耦合一致性、粒子抑制等基础问题制约
了耦合方法成为高保真度模拟方法。
%
\subsection{无网格法(SPH)在流热相变模拟中的进展与不足}
近年来,无网格粒子方法的研究逐渐兴起,为模拟具有混合异质界面和
移动边界的流体流动提供了新的选择。
与传统的基于网格的方法相比,无网格方法由于省去了复杂的网格处理,
并且能够自然地追踪异质界面和移动边界,因此在处理移动界面和剧烈
变形方面具有一些固有的便利性。
在无网格粒子方法中,光滑粒子流体动力学(SPH)方法已成为发展最快、
使用最广泛、最成熟的方法。

无网格粒子方法作为摆脱固定网格约束的连续介质离散策略,与依赖
固定网格结构的方法不同,通过拉格朗日粒子的运动与相互作用来
描述流体和界面行为。
无网格粒子方法如光滑粒子流体动力学(Smoothed Particle 
Hydrodynamics, SPH)和移动粒子半隐式法(Moving Particle
Semi-implicit, MPS),由于其拉格朗日性质逐渐兴起,并在模拟复杂
的多相流方面具有巨大的潜力,为模拟具有混合异质界面和移动边界的
流体提供了新的选择。
在这些方法中,SPH 已成为一种应用广泛且研究深入的方法。

\textcolor{B}{与基于网格的方法相比,SPH是一种拉格朗日粒子方法,
	它无需引入额外的追踪技术即可轻松追踪多相界面。
	这使其在处理多相界面方面更加灵活。因此SPH自然而然地被扩
	展并应用于多相流模拟,并且已经提出了多种多相SPH模型
	\cite{hu2006multi,guo2024smoothed,pozorski2024smoothed,le2025smoothed}。
	在这些模型中,密度计算方案已成为重要的研究点,因为材料的可
	压缩性特征(可压缩和不可压缩行为)直接反映在密度演变中。
	一般来说,多相SPH模拟中处理密度差的方法有两种:密度求和法和
	连续密度法\cite{monaghan2005smoothed}。
	密度求和法被认为更适合处理密度比大的多相流,而连续密度法更适
	合处理密度比小的多相流~\cite{suresh2019comparative}。
	为了精确捕捉密度求和法构建的多相界面处的密度不连续性,
	Hu和Adams~\cite{hu2006multi}从粒子平滑函数推导出了粒子平均
	空间导数的基本近似值。
	在这种方法中,相邻粒子仅对比容有贡献,而对密度没有贡献。
	Park等人~\cite{park2020development}使用参考密度对密度求和进
	行归一化。
	对于连续密度方法,可以使用基于体积的离散梯度来重新表述动量
	方程~\cite{colagrossi2003numerical,grenier2009hamiltonian,
		hammani2020detailed}和扩散项~\cite{tryggvason2001front,
		le2025smoothed, tartakovsky2016smoothed}。
	近年来,黎曼求解器也被用于多相流模拟[3,22]。
	然而,这些方法仅仅从数值角度缓解了由可压缩性差异引起的不稳
	定性问题,而没有从根本上解决核心问题。
	这种局限性源于物理层面上可压缩流体和不可压缩流体的密度演化
	基本方程本质上是不同的。
	近年来,黎曼求解器也被用于多相流模拟~\cite{meng2020multiphase,
		rezavand2020weakly}。
	\textbf{然而,这些方法仅仅从数值角度缓解了由可压缩性差异引
		起的不稳定性问题,而没有从根本上解决核心问题。
		这种局限性源于物理层面上可压缩流体和不可压缩流体的密度演化
		基本方程本质上是不同的。}}

\textcolor{R}{在多相SPH模拟中,不均匀的粒子分布和非物理的
	界面间隙也会导致数值不稳定。
	为了应对这些挑战,人们提出了几种稳定策略。一种被广泛采用的
	策略是引入背景压力,以减轻负压引起的颗粒聚集
	\cite{guo2024smoothed,he2022stable}。}

与传统基于网格的方法相比,由于无网格方法省去了部分复杂的网格
处理,并且能够自然地追踪异质界面和移动边界,因此在处理移动界面
和剧烈变形方面具有一些固有的便利性。
粒子方法也已被用于模拟液-液界面流动[XXX]、两相气-液流动[XXX]、
流体-刚性相互作用[XXX]和流固耦合[XXX]。
由于密度比可忽略不计,凝固相变模拟可以通过增加粘度[XXX]或固定
颗粒[XXX]很容易建模,因此这些粒子法已被广泛应用于分析核工程中
复杂的多相流,包括传热和固液相变。
\textbf{\textcolor{R}{然而,由于密度和体积的显著变化,气液相
		变的建模比固液相变的建模要困难得多。
		开发可靠且通用的汽化或沸腾相变模型可以显著扩展粒子方法的应用范围,
		尤其考虑到沸腾现象在热工程、化学工程和核工程领域中普遍存在。}}
近年来,一些显著的发展显著提高了粒子方法的可靠性,例如高阶离散化
方案[XXX]、输运速度公式[XXX]、粒子移动[XXX]和边界条件改进[XXX]等。

对于粒子方法而言,由于\textbf{\textcolor{B}{①气液界面处密度不
		连续跃变和②气相体积急剧膨胀}},建立沸腾模型尤其具有挑战性。
关于\textbf{\textcolor{R}{密度跃变}},模拟多相气液流动的方法
基本上有两种。
首先是\textbf{\textcolor{R}{同步求解方法}},也称为单流体模型[X]。
在这种方法中,气相和液相被视为一种具有可变密度和粘度的流体。
基本上对所有流体相应用一种离散化模型,最终同时求解。这种方法的
优点是易于实现,但密度突变导致的不稳定性风险是其主要挑战。
Colagrossi和Landrini[X]开发了一种稳定的弱可压缩SPH(WCSPH)
方法,用于水-空气流动,该方法具有改进的人工粘度项和人工内聚力。
随后,基于显式SPH[XXX]、隐式SPH[XX]和隐式MPS[XX]的多相粒子求
解器相继被开发出来。
然而,对于气液两相流,大多数此类方法通常采用密度和粘度平滑
~\cite{zhang2015sph}。
为了处理较大的密度跃变,黎曼求解器也被应用于WCSPH两相流问题
\cite{rezavand2020weakly}。
均匀多相求解器采用了密度约束[X]、均匀平均法[X]和相函数[X]等技术。
与单相模拟相比,该方法获得了更优的结果[XX]。
然而,这类算法在处理较大的体积变化和密度变化以及表示密度和粘度
的急剧不连续性方面存在困难。=
遗憾的是,拉格朗日粒子方法难以处理同时求解过程中较大的体积变化。
密度和黏度的变化规律差异显著,\textbf{密度大跃变和和粘度差异}。
引起极大的不稳定性,控制方程分别对应于可压缩流体和不可压缩流体。
气液两相流算法的稳定性和精度极具挑战性。
这是因为\textbf{不同相之间的压缩性和声速差异显著}。
这些差异显著增加了界面动力学和整体流动行为精确解析的难度。

第二种气液流动方法是\textbf{\textcolor{R}{交替求解方法(或耦
		合算法)}},其中气相和液相的计算(至少是压力计算)交替进行。
这样界面处的不连续密度跃变得以保留,相应的失稳现象也得以避免。
Lind等人~\cite{lind2016incompressible}~最近利用SPH提出了一种
不可压缩-可压缩 SPH 方法,该方法结合了不可压缩SPH和WCSPH方法来
模拟水-空气问题。
耦合策略如下:气相为液相提供狄利克雷压力边界,而液相为气相提供
速度边界。
此外,该方法已与多相PS算法相结合,用于模拟水-空气流
动~\cite{mokos2017multi},同时上述研究中表面张力模型并未被考虑。
然而,正如Nair和Tomar~\cite{nair2019simulations}~指出的那样,
没有考虑在高压梯度液体流动中受限的气泡,也未考虑表面张力模型。
于是他们开发了一种类似的基于SPH方法的可压缩-不可压缩耦合,
其中可压缩-不可压缩相互作用通过他们推导出的Rayleigh-Plesset 
问题得到验证,但在界面附近仍意外地发生了严重的粒子混合,尤其对
于高密度比的情况。
目前文献中仍然缺乏可靠的表面张力气泡流动耦合算法,因此,需要进
一步明确耦合算法中表面张力模型的选择,如Duan等人
\cite{duan2020incompressible}~开发了MPS-SPH方法模拟气泡。
\textbf{\textcolor{R}{由于不同相的交替计算,在第二种方法中
		考虑气体体积膨胀相对容易。}}

模拟沸腾现象的第二个挑战,即\textbf{\textcolor{R}{气体/蒸汽的
		体积膨胀}},将针对粒子方法进行描述。
在拉格朗日框架下,体积膨胀的模拟难以实现。
因此,大多数关于沸腾的数值研究都采用了欧拉网格方法。
气体体积膨胀会导致气相颗粒分布逐渐稀疏。具体而言,气体颗粒半径
会逐渐增大,最终导致界面处出现严重的多分辨率问题。这个问题在同步
求解方法中尤其难以解决。因此,只有少数研究采用粒子方法模拟沸腾相变。
在SPH框架下,
Sigalotti等人~\cite{sigalotti2014diffuse}~基于扩散界面模拟了微
重力环境下范德华(vdW)液滴的蒸发。
为了解决严重的分辨率问题,采用了全自适应SPH模型。范德华状态
方程的内聚部分和毛细力自动产生表面张力,从而维持液滴的形状。
这一方法被进一步扩展以模拟范德华液滴的剧烈沸腾现象~\cite{sigalotti2015smoothed}。
Yang和Kong~\cite{yang2017smoothed}~也开发了一种用于蒸发过程的
SPH方法,该方法允许颗粒质量变化,从而能够轻松计算传质过程。
颗粒质量的变化导致颗粒半径不同;因此,需要开发颗粒分裂/合并技术。
在以上的研究中~\cite{sigalotti2014diffuse, sigalotti2015smoothed,
	yang2017smoothed}, 界面是扩散的而不是清晰的
\textbf{\textcolor{R}{这里要加一个图说明界面的扩散与清晰的对比}}。
此外,Das和Das~\cite{das2015modeling}~提出了一种基于SPH的气液
相变模型,通过引入伪粒子来积累蒸汽质量。
当伪粒子具有足够的质量后,需要进行粒子重分布和变量插值,以考虑体积
膨胀并克服严重的多分辨率问题。
总之,为了处理体积膨胀和相关的多分辨率问题,自适应SPH方法在一定
程度上有所帮助,但会产生扩散界面~\cite{sigalotti2014diffuse, 
	sigalotti2015smoothed,yang2017smoothed}。
在其他情况下,则需要进行粒子分裂/合并~\cite{yang2017smoothed}~
或粒子重分布~\cite{das2015modeling}。
Duan等人~\cite{duan2020incompressible}~开发一种拉格朗日粒子
方法框架,用于处理具有急剧密度跃变的两相气液流动;此外,还试图
证明其能够以直接的方式处理沸腾引起的气体体积膨胀。

\textcolor{B}{在传热和相变领域,SPH引起相当大的关注。
	在传热领域,Cleary和Monaghan~\cite{cleary1999conduction}~是
	第一个将SPH应用于传热问题的人,提出了热传导方程的SPH公式并对
	标准 SPH公式进行了修改,以确保不同材料界面处热流的连续性
	然后,Chen等人~\cite{chen1999corrective}~将修正的SPH方法(CSPM)
	应用于非稳态传热问题,以求解导热模型中核函数的二阶导数,证明了
	其提高边界精度的能力。        
	Jeong 等人~\cite{jeong2003smoothed}~通过将二阶偏微分方程分解
	为两个一阶方程改进了传热算法,使得该方法能够应用于复杂的几何形状。
	他们还实现了幽灵粒子边界,以修改壁面附近的系统变量,从而能够确定
	结构的瞬时温度场。
	Rusty等人~\cite{rook2007modeling}~开发了一种使用Crank-Nicolson
	隐式时间积分的 SPH算法来模拟热传导问题,提高了SPH在长期热传导模拟
	中的稳定性。
	Chaniotis等人~\cite{chaniotis2002remeshed}~提出了一种新的重新网
	格化SPH方法来模拟粘性导热流,准确地考虑了粘度和热扩散效应。
	此外,许多研究人员已将基于 SPH 的热质传递算法应用于自然对流[X]和
	纳米流体动力学[X]等热问题的数值模拟中,进一步证明了其在解决复杂传
	热挑战方面的多功能性和有效性。
	关于热对流,Yang等人[X]进行了自然对流的模拟,并探讨了普朗特数和
	瑞利数对流动状态的影响。
	Zhang等人[X]和Yang等人[X]采用了粒子移位技术(PST)来提高数值模拟
	的精度和稳定性,以模拟高雷诺数和瑞利数下的热对流颗粒流动。
	考虑到相变传热,Monaghan等人~\cite{monaghan2005solidification}
	首先研究了纯液体的凝固过程,包括点热源和不规则系统边界。
	Farrokhpanah等人~\cite{farrokhpanah2017new}~开发了一种新的
	SPH方案,用于研究传热凝固和熔化过程中相变温度附近潜热的变化。}

\textcolor{B}{已有研究极大地促进了SPH传热技术的发展,但针对多
	相流(例如气液流)的传热研究仍然有限,需要进一步研究。
	为了解决这个问题,Fang等人~\cite{fang2025accurate}开发一种精确
	的多相流热力学Riemann-SPH模型。
	该模型考虑了热扩散率对瞬态传热的影响,并引入了黎曼近似来处理不连
	续的温度场,从而与其他SPH方案相比,在热传导模拟方面具有更高的精度。
	此外,通过状态方程将热温度场与压力联系起来,本模型能够考虑热流
	体耦合效应,即流动温度会影响流动的压力场,从而影响流体的运动。
	最后,对于强可压缩流体,通过添加由流体可压缩性引起的内能变化,
	并进一步考虑强可压缩流动中的热辐射效应,传热 SPH 模型得到了
	进一步增强。}

\textbf{\textcolor{B}{孙中国}}
\textcolor{B}{对沸腾和冷凝现象的描述是一个具有挑战性的问题,因为
	其涵盖了各种复杂的现象,例如湍流效应、热传递、相变以及具有大密度比
	的多相流等。
	Sun等人~\cite{han2024incompressible}提出一种拉格朗日不可压缩-可
	压缩无网格粒子方法,用于模拟涉及沸腾和冷凝的多相流。
	该多相方案结合了MPS和SPH,同时引入了基于传热的相变模型来模拟
	气液相变。
	为了获得清晰的相界面并建立相变模型,本研究采用了不进行人工平滑的
	独立时间步进多相方案。}
\textbf{\textcolor{R}{同时,在沸腾和冷凝过程中,气相的体积会发生
		剧烈变化;因此,需要采用粒子分裂和合并方法来防止粒子尺寸出现显著差异。}}
①首要挑战是实现气相和液相的稳定耦合。
因为气相和液相的物理性质差异显著,尤其是在密度、黏度和压缩性方面。
粒子方法通常直接利用粒子模拟多相流。一种有效的方法是平滑物性不连续
性并同步计算各相。
一些研究人员采用单流体模型,即使用物性可变的流体来表示多相流,
并对界面处的物性进行平滑处理。
Lyu等人~\cite{lyu2023towards}~采用连续状态方程计算空化。
平均算子也通常在这些方法中使用来保持连续的加速度场,
比如~\cite{duan2020incompressible,rezavand2020weakly},
然而,这些物理性质平滑方法在处理剧烈的体积变化时并不具有优势。
另一种方法是使用不同的时间步进方法分别计算每个阶段,然后通过边界
件将它们耦合起来~\cite{lind2016incompressible, nair2019simulations, duan2020incompressible}。
\textcolor{B}{②第二个挑战在于相变模型。
	对无网格粒子方法的气液相变模型的研究起步较晚,仍需进一步研究,
	并且欧拉方法中使用的速度散度源项不太适合粒子方法。
	关于粒子法沸腾模型的研究比关于冷凝模型的研究要多。
	Duan等人~\cite{duan2020incompressible}~通过在界面处插入气体颗粒来
	模拟沸腾现象。
	通过引入相应的状态方程,SPH可以将气液两相流的空化、蒸发和沸腾现象
	模拟为连续流体。}

\textbf{\textcolor{R}{几十年来,粒子方法研究界一直致力于研究沸腾
		和冷凝流动这一课题,但尚未建立准确、稳健和统一的纯粒子沸腾-冷凝方案。}}
\textbf{\textcolor{R}{相变怎么产生的比较重要,bubble怎么产生的,
		以及bubble产生之后的密度差该如何处理,体积膨胀该如何处理,两相流
		的问题该如何解决。
		尽管研究对象具有较高数值复杂性,但通过逐级引入物理机制、模块化算法
		设计以及多层次算例验证,可有效控制研究风险,确保研究目标在项目周期
		内实现。}}

The heat transfer process in the hydrogen precooler is dominated by wall-induced subcooled flow boiling under high-pressure forced convection conditions. Local vapor bubble nucleation and condensation occur near the heated wall, while the bulk flow remains in the liquid phase. Smoothed Particle Hydrodynamics (SPH) is adopted to resolve the transient liquid–vapor interface and boiling dynamics at the mesoscale.

现有工作往往把两相耦合、相变源项、多分辨率粒子处理当作可独立叠加的模块,分别改进各自的稳定性或效果;但在液氢/氢气这类强物性差异与体积剧变工况下,这三者在离散层面强耦合——粒子拓扑操作会改变界面几何与离散邻域,进而影响压力/速度通量与界面热通量;相变源项又会改变局部密度/体积分布与压力解的约束,使得“单项优化”容易引入隐性守恒误差与伪流。
%	
\section{发展动态分析与存在的主要问题}
通过以上国内外研究现状分析可见,面向气液界面流动–传热–相变耦合
过程的数值模拟研究总体朝着提升算法鲁棒性与守恒一致性的方向发展。
尽管相关研究取得进展,但针对气液相变两相流的无网格SPH模拟仍存
在共性瓶颈,限制了其在高物性差异、强温差与体积剧变条件下的稳定
性与可靠性。归纳而言,主要问题集中在以下三个方面:

(1)\textbf{\textcolor{B}{界面压力–速度传递离散不相容,导致
		耦合稳定性受限。}}
分离式两相SPH依赖界面边界条件的双向传递实现耦合,但界面核支撑域
截断、粒子扰动及两相可压性差异易放大界面离散误差,使压力–速度
传递离散不相容,诱发界面压力振荡、寄生流与相间掺混等失稳现象。
现有方法多借助平滑或阻尼抑制振荡,但对稳定化强度的选取较为敏感,
难以同时保持稳定性与守恒一致性。

(2)\textbf{\textcolor{B}{相变闭合与潜热交换守恒不足,源项
		易引发耦合误差放大。}}
相变建模中常将潜热以源项直接叠加,或在界面附近插值/平滑更新,
使界面热通量与相变质量通量难以严格闭合,易出现温度过冲、假相变
与能量漂移。
相变引起的质量与能量变化还会改变局部密度与体积分布,进而扰动
压力求解与速度更新,放大耦合误差并诱发相变速率的非物理波动,
限制计算稳定域与适用范围。

(3)\textbf{\textcolor{B}{体积剧变与多时间尺度约束缺少统一
		处理,粒子自适应与守恒难兼顾。}}
液–气相变伴随体积快速变化,造成粒子稀疏与邻域退化;声学CFL与
热扩散/相变源项刚性进一步压缩时间步长,降低推进效率与稳定裕度。
分裂/合并等粒子重构用于维持分辨率,但变量重分配与同步更新缺少
守恒约束,易引入隐性质量/能量误差并扰动压力场,削弱整体稳定性
并限制适用工况。		
%	
\section{本项目拟开展的研究课题}
针对上述瓶颈,本项目拟开展气液相变两相流的SPH守恒一致耦合方法研究,
围绕界面耦合、相变闭合与体积剧变下粒子自适应三个关键环节,在SPH
离散框架内构建守恒一致、稳定可靠的无网格耦合求解方法。

(1)开展界面压力–速度信息传递的离散相容与稳定耦合研究,形成稳健
的界面传递机制,抑制界面压力非物理振荡与寄生流,构建守恒一致的两
相耦合推进框架。

(2)开展热通量驱动的相变质量通量闭合与潜热交换守恒实现研究,提
出与压力求解相协调的相变更新策略,降低温度过冲、假相变与能量漂移。

(3)开展体积剧变条件下的守恒粒子分裂/合并与多时间尺度推进研究,
形成带守恒约束的粒子重构与变量重分配方法,保证拓扑操作不破坏界面
耦合与相变闭合的一致性。
%
\begin{REF}
\subsection*{参考文献}
\vspace{-50pt}
\bibliography{ref}%参考文献
\end{REF}

\newpage%自己判断是否需要
