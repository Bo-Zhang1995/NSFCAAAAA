{\centerline{\Large 气液相变两相流的SPH守恒一致性耦合方法研究}}
\vspace{4pt}
\textbf{摘要}:
%
气液界面流动–传热–相变耦合广泛存在于深冷换热、蒸发冷却与液滴蒸发等
过程。
界面拓扑快速演化、强温度梯度与潜热交换共同作用,呈现多尺度强非线性
特征。
传统网格法受界面重构复杂、潜热离散误差累积与时间步刚性等限制,难以
在强耦合工况下兼顾精度与稳定性;
现有无网格SPH研究多将两相耦合、相变闭合与粒子自适应分离处理,缺乏
统一守恒约束,易引发界面压力振荡、能量漂移与参数敏感。
为此,本项目开展气液相变两相流的SPH守恒一致耦合方法研究:
\textcircled{\raisebox{-0.8pt}{1}}构建界面压力–速度信息传递的
相容离散与稳定边界交换机制,抑制非物理振荡与相间掺混;
\textcircled{\raisebox{-0.8pt}{2}}建立热通量驱动的相变质量通量
闭合与潜热守恒更新方法,实现与耦合推进相容的能量交换;
\textcircled{\raisebox{-0.8pt}{3}}面向液–气相变体积膨胀,发展
守恒约束的粒子分裂/合并与多时间尺度推进,并与界面耦合、相变更新协
同集成。
本项目将通过由简到繁的典型算例逐级验证,形成稳定可靠、可复用的SPH
耦合求解算法与实现流程,为航空航天热防护等液氢深冷换热相变两相流
模拟提供重要的计算方法基础。

\textbf{关键词:} 光滑粒子流体动力学(SPH);无网格方法;
气液相变两相流;守恒一致性耦合;相变闭合

\textbf{Abstract}:


\textbf{Key words:} 
\clearpage	