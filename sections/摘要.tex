%
\noindent\textbf{{
		\centerline{\Large 气液相变两相流的守恒一致性SPH耦合
			方法研究}}}

\vspace{5pt}

\textbf{摘要}:
%
强温差条件下,气液相变界面快速演化,密度跃变与体积变化耦合显著,
呈现多尺度强非线性特征。
传统网格法在强耦合工况下面临界面重构代价高、潜热离散误差易累积与
时间步长受限等问题,难以兼顾计算精度与稳定性;
现有无网格SPH方法多将界面交换、相变更新与粒子拓扑更新分步处理,
界面通量与源项跨时间层进入求解,易诱发界面振荡与能量漂移。
为此,本项目开展气液相变两相流的守恒一致性SPH耦合方法研究:
\textcircled{\raisebox{-0.8pt}{1}}构建界面压力与法向速度的
相容离散与一致交换机制,抑制非物理振荡与相间掺混;
\textcircled{\raisebox{-0.8pt}{2}}建立热通量驱动的相变质量
通量闭合与潜热守恒更新方法,并在同一时间层内一致嵌入两相求解;
\textcircled{\raisebox{-0.8pt}{3}}面向两相压缩性差异与相变体积
剧变,发展多时间尺度同步推进与守恒约束的粒子拓扑更新方法,并与
界面耦合和相变更新协同集成。
通过分层逐级开发与验证,形成可复用的耦合算法与验证方法,
为极端工况下相变两相流的可信预测与性能边界评估提供可验证的数值
工具与方法学支撑。

\textbf{关键词:} 光滑粒子流体动力学(SPH);
气液相变两相流;
守恒一致性;
热力学一致性;
多时间尺度

\textbf{Abstract}:

\textbf{Key words:} 
\clearpage	