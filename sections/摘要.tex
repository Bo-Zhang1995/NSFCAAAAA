%
\noindent\textbf{{
		\centerline{\Large 气液相变两相流的守恒一致性SPH耦合方法研究}}}

\vspace{5pt}

\textbf{摘要}:
%
气液界面流动、传热与相变耦合广泛存在于航天热防护、液氢预冷及深冷储能等
关键工程领域。
%
在强温度梯度与潜热交换作用下,界面快速演化并伴随密度跃变与体积响应,
呈现多尺度强非线性特征。
%
传统网格法受界面重构复杂、潜热离散误差累积与时间步长刚性等限制,难以在
强耦合工况下兼顾精度与稳定性;
%
现有无网格SPH方法多将界面交换、相变更新与粒子演化分步处理,且缺乏温度
相关状态方程下的统一守恒约束,易引发界面振荡与能量漂移。
%
为此,本项目开展气液相变两相流的守恒一致性SPH耦合方法研究:
%
\textcircled{\raisebox{-0.8pt}{1}}在温度相关状态方程条件下,构建界面
压力与速度信息传递的相容离散与稳定交换机制,抑制非物理振荡与相间掺混;
%
\textcircled{\raisebox{-0.8pt}{2}}建立基于界面能量平衡的相变速率计算
与潜热守恒更新方法,控制守恒偏差并与两相推进相容;
%
\textcircled{\raisebox{-0.8pt}{3}}面向相变体积剧变,发展守恒约束的
粒子分裂与合并及多时间尺度推进策略,并与界面耦合、相变更新协同集成。
%
本项目将通过分层算例逐级验证,形成稳定可靠、可复现且可复用的SPH耦合求解
算法与流程,为相关装备与部件的性能评估与优化设计提供关键方法基础。

\textbf{关键词:} 光滑粒子流体动力学(SPH);
气液相变;
温度相关状态方程;
守恒一致性;多时间尺度

\textbf{Abstract}:


\textbf{Key words:} 
\clearpage	