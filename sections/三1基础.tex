\subsection{研究基础}
申请人长期从事光滑粒子流体动力学(SPH)方法的关键
数值算法研究与相应开源求解器开发,作为主要成员参与
德国科学基金(Deutsche Forschungsgemeinschaft, DFG)
等相关项目,并持续建设与维护开源求解器SPHinXsys。
围绕SPH守恒高阶一致性算子构造、复杂自由表面/强瞬态
载荷流固耦合等问题的基准检验与工程化实现,申请人已
形成较为系统的研究积累与实现经验。主要包括以下几个
方面的相关工作基础:

\textbf{(1)守恒高阶一致性离散与应用工作基础}

申请人围绕核函数梯度修正、核函数设计与守恒高阶近似
等关键环节开展研究,并在开源平台SPHinXsys中实现
相应算法模块与多算例对比验证。

一方面,针对传统守恒SPH在梯度离散中易出现一致性
不足、误差累积以及数值耗散偏大的问题,申请人提出
在保持守恒形式的前提下,引入反向核梯度修正并结合
修正矩阵驱动的粒子松弛策略,显著提升了离散算子的
一致性与误差控制能力,将守恒形式离散提升至二阶
收敛;在内流与自由表面流等典型场景中的系统验证均
表现出更好的数值精度与收敛特性,尤其是解决了自由
表面流模拟中长期存在的过度耗散问题。
驻波、波浪能量转换装置等典型2D/3D算例结果如图
\ref{守恒高阶一致性}所示。
相关成果发表在计算力学领域顶级期刊\textbf{Computer 
	Methods in Applied Mechanics and 
	Engineering(2025, 433:117484)}。
	
%
\begin{figure}[h!]
	\centering %图片居中
	\includegraphics[width=14.7cm]
	{figures/守恒一致性图}
	\captionsetup{justification=centering} 
	\caption{守恒高阶一致性工作基础
		(上:驻波算例的能量耗散与压力云图;\\
		下:波浪能量转换装置算例的定量对比与速度云图)}
	\label{守恒高阶一致性}
\end{figure}
%	

另一方面,面向主流核函数在低分辨率下精度提升受限
的问题,申请人参与提出四阶截断Laguerre–Gauss核函数,
并在一系列2D/3D流体与固体力学算例中展开系统验证,部分
结果如图\ref{高阶核函数}所示。
结果表明:该核函数在保持紧支域大小与计算效率的同时,
可有效改善粒子松弛质量与高阶近似能力,使欧拉SPH与完全
拉格朗日SPH在同等分辨率下均呈现出更快的收敛趋势与更低
的误差水平。
相关成果发表在计算流体力学领域顶级期刊\textbf{
	Journal of Computational Physics
	(2024, 519:113385)}。
	
%
\begin{figure}[h!]
	\centering %图片居中
	\includegraphics[width=14.7cm]
	{figures/高阶核函数}
	\captionsetup{justification=centering} 
	\caption{高阶核函数构造与验证工作基础
	(左:三维弯曲柱应力云图及收敛对比;\\
	右上:顶盖驱动半圆腔体速度云图;
	右下:三维振荡板应力云图)}
	\label{高阶核函数}
\end{figure}
%
	
此外,面向强冲击载荷下的多分辨率流固耦合问题,
申请人将一致性修正与稳健通量框架Riemann-SPH进行融合
改进。
通过在守恒框架内引入与多分辨率匹配的通量传递与修正
策略,增强了冲击信息在流体与结构界面附近的稳定传递
能力,提升了自由表面形态与结构响应预测的可靠性。
在溃坝冲击、晃荡箱体等典型流固耦合算例中与实验观测
取得良好一致性,结果如图\ref{修正黎曼流固耦合}所示。
相关成果发表在流体力学知名期刊\textbf{Physics 
	of Fluids(2025, 37:042122)}。
	
%
\begin{figure}[h!]
	\centering %图片居中
	\includegraphics[width=14.7cm]{figures/修正黎曼流固耦合}
	\captionsetup{justification=centering} %图题居中
	\caption{修正黎曼多分辨率流固耦合工作基础
		(上:溃坝水流冲击弹性板的结果验证;\\
		下:带弹性挡板滚动水箱晃动算例的结果验证)}
	\label{修正黎曼流固耦合}
\end{figure}
%
	
上述工作为本项目后续界面耦合算子的守恒一致性构造
及数值验证提供了可直接复用的高阶离散基础。

\textbf{(2)传热与能量建模与工程验证相关工作基础}

围绕传热过程的求解与工程可用性,申请人开展了SPH
框架下热传导方程的稳定数值求解研究,并形成了从模型
实现到定量评估的完整经验。
针对热传导在强温度梯度与复杂热边界条件下易出现时间
步长受限、迭代效率低与温度场振荡等问题,构建了隐式
分裂时间推进求解策略,提高了热传导计算在较大时间步
长下的稳定性与推进效率,使温度场演化保持稳定收敛且
误差可控。
在此基础上,申请人将该传热求解能力嵌入到目标驱动的
迭代计算流程中,开展工程导向的参数/分布反演与优化
分析,进一步检验了该求解策略在不同热边界条件与多类
工况下的鲁棒性与可扩展性,并积累了热边界施加、热
通量与温度场后处理、误差评估等关键经验。
复杂热源及边界条件下的计算结果如图\ref{隐式分裂求解}
所示。
相关成果发表在传热与传质领域知名期刊
\textbf{International Journal of Heat and Mass 
	Transfer(2024, 227:125476)}。
上述工作为本项目后续界面热通量评估与能量方程稳定
推进提供了成熟的建模与验证经验。

%
\begin{figure}[h!]
	\centering %图片居中
	\includegraphics[width=14.7cm]
	{figures/传热优化图}
	\captionsetup{justification=centering}
	\caption{传热与能量建模与工程验证工作基础
	(左:热边界条件示意;\\中:均匀热导率下的温度场;
	  右:优化热导率分布下的温度场)}
	\label{隐式分裂求解}
\end{figure}
%

\textbf{(3)粒子松弛与多分辨率耦合相关工作基础}

面向无网格粒子方法中普遍存在的粒子无序与分辨率不
一致等数值风险,申请人开展了粒子松弛与多分辨率耦
合计算研究,形成了可迁移的粒子层稳定控制方法与工
程实现经验。
一方面,提出并实现了以物理约束与数值稳定性为导向的
复杂几何粒子松弛与质量分布控制策略,并通过隐式分裂
求解提升了松弛过程的稳健性与效率,使粒子在长时间
推进、强变形及自由界面剧烈演化条件下保持分布均匀、
从而降低由粒子无序引发的梯度评估偏差与压力噪声;
隐式粒子松弛收敛及复杂几何粒子生成如图\ref{隐式粒子松弛}
所示。
相关成果发表在知名期刊\textbf{Engineering Analysis 
	with Boundary Elements(176 (2025) 106239)。}
另一方面,针对工程计算中不可避免的局部加密需求,
建立了多分辨率建模与耦合求解的实现流程,贯通离散构造、
跨分辨率信息交换与典型算例对比验证,为跨尺度计算提供
了稳定可行的工程化路径。
相关成果发表于知名期刊\textbf{Physics of Fluids
	(2025, 37:042122)}。	
上述工作为本项目在邻域退化与体积剧变条件下的粒子
层稳定控制与多尺度推进提供了工作基础。

%
\begin{figure}[h!]
	\centering %图片居中
	\includegraphics[width=14.7cm]
	{figures/隐式粒子松弛}
	\captionsetup{justification=centering} 
	\caption{粒子松弛工作基础
	(上:隐式粒子松弛收敛特性;下:复杂几何粒子生成)}
	\label{隐式粒子松弛}
\end{figure}
%

\textbf{(4)开源平台SPHinXsys的工程化开发与实现
	相关工作基础}
	
申请人作为主要开发人员长期参与并持续维护基于SPH方法
的多物理场仿真开源求解器SPHinXsys,在开源工程环境下
形成了面向数值算法研发的工程化组织与实现能力。	
围绕算法快速集成、算例对照评估与结果追溯,申请人参与
构建并完善了平台的模块化框架与配套工具链。
形成了统一接口下的流体、结构与传热等多物理模块组织
方式,支撑新算法按组件化方式接入与迭代;
搭建了典型示例算例库与标准化验证流程,便于不同物理
模块及耦合场景的横向对比;
建立了自动化回归测试与持续验证机制,实现版本演进过程
中的一致性监测与质量控制,保障平台长期维护与工程可用性。
上述平台测试相关流程如图\ref{回归测试}所示,相关
成果发表在知名期刊\textbf{Journal of Hydrodynamics
	(2024, 36(3):466–478)}。
依托该平台,申请人形成方法提出、代码实现与基准对照验证
的稳定工作范式,为本项目后续算法集成、算例复现与系统
化对比验证提供可靠的软件平台与工程条件支撑。

%
\begin{figure}[h!]
	\centering %图片居中
	\includegraphics[width=14.7cm]
	{figures/回归测试}
	\captionsetup{justification=centering} 
	\caption{SPHinXsys平台回归测试体系工作基础}
	\label{回归测试}
\end{figure}
%

综合以上研究积累,申请人在守恒高阶离散、强瞬态耦合问题
的稳健求解、传热过程的稳定推进与定量评估、粒子层稳定控
制及多分辨率工程实现等方面已具备扎实的研究基础;
同时依托SPHinXsys平台形成了较成熟的算法实现与验证体系。
上述基础将有效支撑本项目后续关键算法的快速落地与规范化
对照验证,使研究工作具备明确的可操作路径与良好的
实施可行性。

\subsection{可行性分析}
本项目的可行性主要依赖于申请人及所在团队具有支撑关键算法
实现与可信验证的研究积累。
依托既有SPH关键算法基础、持续建设的SPHinXsys求解平台与
长期形成的检验经验,本项目能够围绕气液相变两相流关键算子
与多场更新迭代流程开展可验证的研究,并支撑工程应用。

\textbf{(1)基于既有SPH框架的两相/相变耦合计算}

两相相变SPH模拟的核心在于相变源项引入后的守恒闭合、物性
跃变处界面信息交换的稳定性,以及耦合更新顺序对误差传播
的影响可控。
申请人长期从事SPH关键数值算法研究与工程化实现,在守恒
一致性与高阶算子构造方面已形成较为成熟的方法体系;
同时,持续维护的求解器平台已具备自由表面、多相界面处理、
边界条件施加、双准则时间步推进与粒子管理等通用能力,
为本项目方法落地提供稳定载体。
在此基础上,本项目的工作重点是围绕守恒一致性要求重构
相变闭合与热/质/动量更新流程,明确界面交换与两相推进算子
匹配关系,并将关键约束固化为可复现实现与可检查的守恒核算,
耦合结构与更新机制基于已有工作可稳健开展。

\textbf{(2)分层次稳定性验证与对比评估计算体系}

相变两相问题计算的可信度需要由定性与定量对比的证据支撑。
本项目将采用逐级约束的基准对比流程,首先在具备解析/半解析
参照的相变基准上建立能量收支核算与误差收敛检查,确保相变
源项与能量闭合经得起检验;
随后在界面强变形与拓扑演化算例中,以压力振荡水平、相间掺混
与守恒残差等指标刻画界面交换的稳定性与敏感性;
最后在沸腾气泡生长/并和、避免相变等综合场景中检验多时间尺度
约束下的整体鲁棒性与效率。
依托现有代码体系中成熟完善的监控与可视化模块,上述算例可
固化为回归测试集,使每一次方法迭代都能通过指标变化获得
明确归因,形成可复核的可性证据,并制成按阶段验收。

\textbf{(3)关键风险可定位及可回退的工程化应对措施与支撑条件}

本项目风险主要集中在三类触发源,并均具备明确的控制抓手与回退路径。
其一,高密度比与界面邻域退化可能诱发压力振荡与寄生流;对此通过
对界面交换施加一致性与稳定性约束,并以伪流幅值与守恒残差作为
硬指标进行稳定域标定,确保不稳定现象可量化、可抑制。
其二,相变源项与推进过程的时序不匹配可能导致能量不守恒与温度
非物理波动;对此以能量收支核算作为首要门槛,将相变更新与守恒
更新纳入一致的时间层对齐流程,并保留更稳健闭合形式与更保守时间
步作为回退,优先保证可信性。
其三,相变体积剧变可能造成粒子分布退化并引发多时间尺度冲突;
对此将粒子质量维护与时间推进协同组织,以界面质量指标与守恒
残差触发自适应调整,保证复杂工况下的连续推进能力。
此外,既有平台与计算条件可支撑必要的参数扫描与对比评估,为
稳定域标定与不确定性收敛提供保障。

综上,本项目以既有方法学积累与可复现平台为支撑,把创新集中落
在关键耦合算子与更新流程的可验证实现上,并通过分层基准与硬指
标约束形成可复核的可信性证据链;主要风险具备可执行的控制与
回退路径。
因此,项目实施条件充分、过程可验收,预期可按阶段形成可复现的
算法与算例成果。




