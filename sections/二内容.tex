%
\section{研究目标与关键科学问题}
\subsection{总体研究目标}
本项目面向深冷与高热流密度相变换热等工况对气液相变两相流数值
模拟的需求,开展基于无网格SPH的守恒一致性耦合算法研究。
在强温差、高密度比与体积剧变并存条件下,数值模拟同时面
临\hl{界面一致交换}、\hl{相变守恒更新}与\hl{同步守恒推进}三
方面挑战;围绕这些挑战,本项目将重点解决\hl{强物性跃变界面处
动量与热质通量的一致交换及守恒闭合}和\hl{多时间尺度与拓扑演化
条件下的同步守恒及稳定推进}两大核心科学问题。
总体研究思路如图\ref{总体研究思路}所示。

\begin{figure}[h!]
	\centering %图片居中
	\includegraphics[width=14.7cm]{figures/总体研究思路}
	\captionsetup{justification=centering} %图题居中
	\caption{本项目总体研究思路示意图}\label{总体研究思路}
\end{figure}

本项目属于无网格法(SPH)数值算法的基础研究,主要研究目标包括:

(1)\hl{界面一致交换}:构建与两相推进相协调的界面压力边界重
构与速度约束施加的一致性方法,在高密度比与支撑域截断等条件下
实现压力与速度信息的稳定传递,保持界面清晰度与抑制压力振荡。

(2)\hl{相变守恒更新}:建立热通量驱动的相变质量通量闭合与潜热
守恒更新方法,明确相变更新与两相约束求解的协调关系,降低温度过
冲、假相变与能量漂移,提高相变速率计算的稳定性与准确性。

(3)\hl{同步守恒推进}:建立体积剧变条件下带守恒约束的粒子分裂
合并与变量重分配策略,形成与多时间尺度推进匹配的时间步同步更新
机制,在粒子重构与多时间尺度更新过程中抑制守恒误差累积。

在上述目标基础上,将形成一套面向气液相变两相流的分离式SPH守恒
一致性耦合求解方法,实现界面耦合、相变更新与体积剧变处理在统一流
程内协同推进,并通过由简到繁的典型算例开展定量评估与系统检验。

\subsection{拟解决的关键科学问题}
围绕\ct{强温差、高密度比与体积剧变}并存的气液相变强耦合过程,
本项目拟解决以下两条关键科学问题:

\textbf{(1)\textcolor{B}{强物性跃变界面处动量与热质通量的
		一致交换及守恒闭合机制}}
	
气液相变中,界面既传递压力与速度的相间作用,也承载由热通量驱动
的质量交换与潜热收支。
分离式两相SPH由于两相可压性模型与推进方式不同,界面压力与速度
耦合、相变闭合与潜热更新常被分别处理。
在高密度比与支撑域截断条件下,这种处理方式易导致界面通量表达与
交换不一致、质量与能量收支难以保持一致守恒,进而诱发非物理流动、
相间掺混、温度过冲、假相变与能量漂移等失稳现象。
其科学本质是在界面几何与核加权一致的前提下,动量交换与热质闭合需
采用一致的通量离散定义,并与两相约束求解相容,从而避免界面误差
在耦合推进中被持续放大。
本项目将据此建立可检验的界面离散约束与交换机制,为强耦合求解提
供可靠的界面耦合基础。

\textbf{(2)\textcolor{B}{多时间尺度与拓扑更新条件下的稳定
		推进与同步守恒机制}}

相变引起的质量转移与体积变化会快速改变粒子尺度与邻域结构,并通
过状态方程与压力约束反馈到速度更新。
同时存在显著的稳定步长限制与强源项约束,使计算通常依赖分步更新、
多速率推进以及粒子分裂合并等操作以维持可计算性。
但在界面邻域退化与强源项并存时,时间层不同步会使误差跨步传递并
累积,表现为压力噪声放大、相变速率非物理波动以及守恒误差累积,
从而收缩可稳定计算的范围。
其科学本质是多速率推进与拓扑更新并存时,界面交换与相变更新需在
同一时间层完成同步更新,并在分裂合并与变量重分配过程中维持守恒
约束,从而抑制跨步误差累积。
本项目将以时间层同步与守恒一致性为核心约束,构建体积剧变等场景
下的稳定推进框架。			

\section{研究内容}
为达到上述目标,本项目拟开展三个方面的研究工作。研究内容及
其相互支撑关系如图\ref{研究内容}所示。具体研究内容阐述如下:

\begin{figure}[h!]
	\centering %图片居中
	\includegraphics[width=14.7cm]{figures/研究内容}
	\captionsetup{justification=centering} %图题居中
	\caption{本项目拟开展的研究内容}\label{研究内容}
\end{figure}

\textbf{\textcolor{B}{研究内容1:强物性跃变界面动量交换的一致
		离散与稳定耦合方法}}
\begin{enumerate}
	
	\item[] \textbf{(1)界面压力边界与速度约束的统一离散表述:}
	面向分离式两相SPH,针对两相可压性模型与推进方式差异,在受
	限核支撑与界面几何约束下,构造压力边界重构与速度约束施加的
	统一离散表达,使其共享一致的界面几何信息与离散权重,降低由
	界面条件交换不一致引入的误差。
	
	\item[] \textbf{(2)邻域退化下的误差抑制与动量通量平衡:}
	针对核支撑域截断与粒子扰动导致的界面误差放大,研究与上述界
	面离散形式相匹配的稳定化处理与强度选取方法,并将表面张力等
	代表性界面力项纳入同一权重体系用于动量平衡检验,抑制压力振
	荡、非物理流动与相间掺混。
	
	\item[] \textbf{(3)动量交换的离散相容条件与闭合约束:}
	建立界面压力边界重构、速度约束施加与两相动量更新算子之间的
	离散匹配关系,明确界面动量交换在离散层面应满足的守恒与平衡
	要求,使界面作用不会在耦合推进中持续注入非物理动量注入,从
	而避免界面误差在耦合更新中被持续放大。

\end{enumerate}


\textbf{\textcolor{B}{研究内容2:热通量驱动的相变闭合离散与
		潜热守恒更新方法}}
\begin{enumerate}
	
	\item[] \textbf{(1)物性跃变条件下界面热通量的稳定计算:}
	针对强温差与物性突变使界面温度梯度估计易失真的问题,研究能
	量方程在SPH框架下的离散形式与界面处理方法,构造与界面几何
	信息一致的热通量计算方式,抑制界面附近温度的数值振荡,为
	相变计算提供可靠的热通量输入。
	
	\item[] \textbf{(2)相变质量通量闭合与潜热守恒更新:}
	基于界面能量平衡关系,建立由界面热通量确定相变质量通量的闭
	合表达,在离散层面实现质量交换与潜热吸放的守恒更新,抑制温
	度过冲与能量漂移;同时明确密度、体积与状态量的更新顺序,
	避免潜热重复计入与能量误差累积。

	\item[] \textbf{(3)相变能量源项与两相耦合求解的一致嵌入:}
	明确相变质量与能量更新进入压力求解、速度更新与界面交换的离
	散处理方式,给出与两相约束求解相协调的能量源项分配与变量更
	新方式,保证相变更新不破坏界面动量交换与能量收支的一致性,
	避免引入额外压力噪声与非物理速度。

\end{enumerate}

\textbf{\textcolor{B}{研究内容3:体积剧变下的多时间尺度推进
		与粒子拓扑更新约束}}
\begin{enumerate}
	
	\item[] \textbf{(1)多时间尺度推进组织与跨相同步更新:}	
	面向两相可压性差异与相变源项引入的稳定步长差异,研究多时间
	尺度推进的组织方式,明确液相大步与气相子步的同步更新规则;
	规定同步节点处界面交换量的一致更新要求,减少时间步不同步带
	来的时序误差与耦合偏移。
	
	\item[] \textbf{(2)拓扑更新的时空安排与邻域质量维持:}
	针对体积膨胀或收缩导致的粒子尺度变化与邻域退化,研究分裂与
	合并等拓扑更新的执行时机与作用范围;重点约束界面邻域与相变
	活跃区的粒子分布与核支撑完整性,避免拓扑变化引起界面几何量
	失真及通量离散精度下降。
	
	\item[] \textbf{(3)拓扑更新下的守恒重分配与耦合约束:}
	构建拓扑更新前后质量、动量与能量守恒一致的变量重分配方法,
	并与研究内容1的界面动量交换与研究内容2的相变质量与潜热更
	新保持一致的离散表达;明确拓扑更新在多时间尺度同步推进中
	的嵌入原则与必要的约束条件,抑制由重分配引入的守恒误差与
	压力扰动在多步推进中的累积放大。
	
\end{enumerate}

\textbf{\textcolor{B}{研究内容间支撑关系:}}
三项研究以守恒一致更新为主线,相互支撑与耦合,其关系如图
\ref{研究内容}所示。
研究内容1建立强跃变界面的压力速度一致离散与稳健动量交换,为研究
内容2提供可靠的界面条件;
研究内容2实现热通量驱动的相变闭合与潜热守恒,其质量与能量变
化将反向作用于界面并诱发体积剧变;
研究内容3通过多时间尺度同步推进与拓扑更新的守恒约束,保障界
面交换与相变更新在多尺度时间步长与重构过程中保持一致与稳定。

\section{拟采取的研究方案}

\subsection{技术路线}
本项目研究遵循图\ref{技术路线}所示技术路线。
面向强物性跃变与相变诱导体积剧变条件下气液相变两相流的高保真
数值模拟需求,
以\hl{分离式SPH守恒一致性耦合框架为主线},基于申请人及所在
团队前期开发的\hl{开源SPH求解平台SPHinXsys为实现与验证载体},
围绕研究内容依次开展界面动量一致交换、相变守恒更新与多时间尺度
守恒推进三类关键模块的构造与一致耦合集成。
整体路线以模块内一致构造、模块间一致耦合、结果可一致检验为总
体原则,避免将界面处理、相变更新与多时间尺度推进作为相互独立
的叠加步骤。

\begin{figure}[h!]
	\centering %图片居中
	\includegraphics[width=14.7cm]{figures/技术路线}
	\captionsetup{justification=centering} %图题居中
	\caption{本项目拟采用的技术路线}\label{技术路线}
\end{figure}

针对研究内容一,形成强跃变界面动量交换的统一离散与稳定耦合
方法,统一界面压力边界重构与速度约束施加的离散形式及执行规
则,为相变更新提供可靠界面条件。
在此基础上,建立界面热通量的稳定计算方法,构造热通量驱动的
相变质量通量闭合与潜热守恒更新,并将相变质量与能量更新一致
嵌入两相求解流程,实现相变更新与界面动量交换的守恒一致与协
调推进。
进一步面向体积剧变导致的粒子尺度变化与邻域退化,建立多时间
尺度推进的跨相同步更新规则,提出粒子分裂合并与变量重分配的
守恒约束,明确拓扑更新在同步节点处的嵌入位置与约束条件,保
证拓扑更新与多速率推进不破坏前两阶段的界面交换与相变更新一
致性。

验证采用逐层递进方式组织,并在SPHinXsys平台内沉淀为可复现
的基准算例套件:先检验非相变界面动量交换的稳定性与守恒性,
再检验相变热质基准中的闭合与能量收支一致性,最后在体积剧变
与强耦合综合场景中评估多时间尺度推进与拓扑更新约束下的整体
稳定性与守恒误差水平。

\subsection{研究方法}
结合技术路线图,围绕三项研究内容的具体的研究方法阐述如下:

\ct{(1)}\hl{界面接触状态驱动的压力速度一致交换方法(针对研究内容1)}

分离式两相SPH中,气相与液相采用不同可压性模型与推进方案,界面
处往往分别施加压力边界与速度约束;
当两类边界由不同离散表达与核加权方式给出时,界面误差将以不同形
式反复注入压力约束与速度更新,诱发虚假流动、压力振荡与相间掺混。
在SPHinXsys多相作用与边界处理框架的基础上,本项目拟以\ct{界
面接触态}作为唯一交换对象,如图\ref{研究方法1}所示,在每一时间
步仅构造一对界面量$(p^{\ast},u_n^{\ast})$,并要求其在两相推
进中\ct{同源构造、同核加权调用、同时间步交换};同时将界面动量
交换写成\ct{守恒型成对通量},使跨相作用以成对反
对称形式进入两相动量更新,从离散结构上避免界面作用成为隐含动量
源。
具体而言,在界面法向上采用两材料声学近似的黎曼接触态,仅对界面
共享变量$(p,u_n)$求接触解,并以等效声阻抗$Z=\rho c_{0}$定义
权重,使界面动量交换遵循同一通量离散表达:
\begin{equation}
	\begin{cases}
		p^{\ast}=\displaystyle\frac{Z_g\,p_l+Z_l\,p_g+Z_l 
		Z_g\,(u_{n,l}-u_{n,g})}{Z_l+Z_g}\\[2mm]
		u_n^{\ast}=\displaystyle\frac{Z_l\,u_{n,l}+Z_g\,
		u_{n,g}+(p_l-p_g)}{Z_l+Z_g}
	\end{cases},
\end{equation}
其中下标$l,g$分别表示液相与气相,$c_0$为与相内可压性模型或压
力约束强度一致的等效\ct{特征速度}。
推进组织上,液相压力约束在界面处采用$p^{\ast}$提供压力信息,
气相显式动量更新采用$u_n^{\ast}$提供法向速度条件;二者均在同
一时间层完成交换。
界面动量交换采用\ct{成对反对称守恒型}通量形式进入两相更新,
并在包含表面张力时将曲率压力项并入同一通量离散以检验静力平衡,
从而避免界面力以额外修补项方式破坏守恒结构。

\begin{figure}[h!]
	\centering %图片居中
	\includegraphics[width=14.7cm]
	{figures/研究方法1}
	\captionsetup{justification=centering} 
	\caption{本项目拟采用的两相耦合方法示意图}
	\label{研究方法1}
\end{figure}

针对界面邻域不全与退化,将\ct{核归一残差}与\ct{核矩条件}等可
计算的邻域质量指标作为触发准则,在接触态构造与通量计算中采用
\ct{反向守恒核梯度修正},并结合\ct{速度输运公式}等必要的局部
粒子控制技术以恢复支撑完整性;
进一步将界面带成对作用力平衡与通量一致性交换作为两相动量更新算
子之间的硬约束,构造界面动量闭合残差指标,并采用不破坏守恒结构
的最小扰动修正,使残差维持在可控范围,从而抑制界面误差在耦合推
进中的传播放大。

验证将按照由静到动、由弱到强的分层体系,分别检验静力与界面力一
致性、动态界面下的稳定性、以及强变形条件下的稳定域;统一采用守
恒一致性误差、静力平衡误差、界面清晰度与动量守恒误差等指标进行
定量评估。

\ct{(2)}\hl{热通量驱动的相变闭合与潜热一致更新方法(针对研究内容2)}

在强温差与物性跃变条件下,界面热通量计算易受温度间断与邻域退化
影响而产生噪声,进而诱发温度过冲与能量漂移。
依托SPHinXsys能量方程求解框架,本项目拟采用\ct{热通量稳定计算、
质通量闭合与潜热守恒更新}的统一流程。如图\ref{研究方法2}所示,
在界面带内构造可控的法向热通量,并据此闭合相变质量通量与潜热收
支,使相变更新在离散层面满足质量与能量守恒,并与两相耦合求解保
持一致协调。
具体而言,在界面法向构造热接触态$(T^{\ast},q_{n,l}^{\ast},
q_{n,g}^{\ast})$用于热通量稳定计算。
为适配跨相热扩散率比与瞬态热边界层尺度差异,引入以\ct{热渗透率}
$E$为权重的接触温度
\begin{equation}
	T^{\ast}=\displaystyle\frac{E_l\,T_l+E_g\,T_g}{E_l+E_g},
	\qquad 
	E=\displaystyle\frac{k}{\sqrt{\alpha}}=\sqrt{k\rho c_p},
\end{equation}
并在界面两侧分别给出法向热通量
\begin{equation}
	q_{n,l}^{\ast}=-k_l\,\displaystyle\frac{T_l-T^{\ast}}{\Delta n_l},\qquad
	q_{n,g}^{\ast}=-k_g\,\displaystyle\frac{T^{\ast}-T_g}{\Delta n_g},
\end{equation}
其中$\Delta n_l,\Delta n_g$由界面几何与邻域尺度确定。
相变质量通量按界面能量平衡闭合
\begin{equation}
	\dot{m}=\frac{q_{n,l}^{\ast}-q_{n,g}^{\ast}}{h_{lv}},
\end{equation}
并以\ct{成对反对称守恒型}同步更新两相质量与潜热收支,满足
$\Delta m_l+\Delta m_g=0$,同时潜热项以$\pm \dot{m}h_{lv}$进
入两相能量更新以闭合显热与潜热收支。
为抑制假相变与温度过冲,设置基于饱和温度与可用能量的限制规则,并
以界面有效面积$A_{eff}$权重分配相变质量通量$\dot{m}$,避免单步
潜热吸放超过可用热量。

\begin{figure}[h!]
	\centering
	\includegraphics[width=14.7cm]{figures/研究方法2}
	\captionsetup{justification=centering}
	\caption{本项目拟采用的热接触态驱动的热通量计算、
		相变闭合与源项嵌入流程示意}
	\label{研究方法2}
\end{figure}

将由$\dot{m}$诱发的质量与体积变化作为\ct{同一时间层源项}并入压
力与速度求解以及界面交换过程。
在同步时间层由热接触态给出$q_{n,\cdot}^{\ast}$与$\dot{m}$并完
成守恒更新,随后将体积源项一致嵌入两相约束与状态量更新,并与研究
方法1的界面交换保持同步闭合。
必要时在单步内进行一次校正以抑制源项引起的耦合偏移与压力噪声。
在$T^{\ast}$与$q_{n,\cdot}^{\ast}$的构造及通量计算中全局采用
\ct{反向守恒核梯度修正},且依托\ct{隐式粒子分裂}方法并结合\ct{守
恒投影构造}求解能量方程以提高可用时间步长与计算收敛速度。

验证将分别检验热通量计算的稳定性与收敛性、相变闭合下质量与能量的守
恒一致性,以及源项一致嵌入后的整体稳定域;采用能量守恒误差、温度过
冲幅值、相变速率波动与界面位置等指标进行定量评估。

\ct{(3)}\hl{多时间尺度同步推进与守恒约束拓扑更新方法(针对研究内容3)}

相变引起的体积膨胀与收缩会快速改变粒子尺度与邻域结构,并与两相可压
性差异叠加,形成显著的稳定步长限制,计算需采用多时间尺度推进并配合
粒子拓扑更新。以SPHinXsys的双准则时间步推进与粒子管理机制为基础,
如图\ref{研究方法3}所示,本项目拟以\ct{同步节点}组织跨相多速率推进,
并以\ct{守恒约束下的最小扰动重构}规范分裂合并与变量重分配,从推进
结构上抑制跨步误差累积。
时间推进采用液相主步与气相子步的多速率框架:在同步节点处完成一次跨
相一致更新,将界面动量交换与相变闭合置于同一时间层;子步阶段仅进行
相内推进,并以时间一致的预测提供界面边界量。
同步节点处检查界面带与相变活跃区的邻域质量,以粒子尺度差、核支撑完
整性与核归一残差等可计算指标作为触发准则,确定拓扑更新的作用范围与
强度;拓扑操作限定在同步节点或其邻近窗口,避免在压力约束求解过程中
插入结构突变。

\begin{figure}[h!]
	\centering
	\includegraphics[width=14.7cm]{figures/研究方法3}
	\captionsetup{justification=centering}
	\caption{本项目拟采用的多速率推进的同步节点组织与拓扑更新嵌
		入位置示意}
	\label{研究方法3}
\end{figure}

拓扑更新采用\ct{尺度驱动}的\ct{守恒约束下最小扰动重构}完成变量重分配:
分裂与合并的数目由两相尺度差异与邻域质量自适应确定,避免采用固定阈值下
的触发即分裂合并方法;
重构在满足质量、动量与能量守恒的同时,以同步节点的目标尺度场为参照,在
受影响邻域内结合\ct{粒子各向同性松弛技术}最小化对密度分布与核矩一致性
的扰动,使拓扑变化既不引入隐含守恒源,也不放大邻域退化,从而与多速率
同步推进保持一致。其典型形式可写为
\begin{equation}
	\begin{aligned}
		\min_{\{x_k,m_k\}}\;& \sum_{i\in\Omega}\big(\rho_i^{\,new}-\rho_i^{\,tar}\big)^2,\quad
		\text{约束:}
		\left\{
		\begin{aligned}
			&\sum_k m_k=m_0\\
			&\sum_k m_k u_k=m_0 u_0\\
			&\sum_k m_k e_k=m_0 e_0
		\end{aligned}
		\right.
	\end{aligned}
\end{equation}
其中$\Omega$为受影响邻域,$\rho^{tar}$为同步节点处的目标密度与尺度场。
上述约束保证拓扑更新不引入隐含守恒源,并将邻域质量变化控制在可控范围内,
从而与研究方法1和2的界面交换与相变更新在同一时间层保持一致。

验证将聚焦强耦合与体积剧变场景,选取受限通道内相变诱发的泡团增长并合与
演化等综合算例进行定量评估;统一采用质量、动量与能量守恒误差,以及流动
振荡强度与相变速率波动等指标,衡量稳定域与误差累积水平。

\subsection{关键技术}

\ct{(1)}\hl{界面交换量的统一定义与守恒闭合约束}

强物性跃变界面同时涉及动量传递与热质交换。若压力边界、速度约束与热通量在
不同离散表达下分别给出,界面误差会以不同形式反复进入求解过程,诱发压力振
荡、非物理流动及能量漂移。
本项目的关键技术是建立\ct{统一的界面交换量}:以同一组界面量描述压速信息
与热质信息的跨相传递,使界面几何、核加权与邻域统计在两相中一致调用;并将
跨相作用纳入\ct{守恒闭合约束},要求界面带的动量与能量交换满足成对平衡与
收支一致,从结构上避免界面处理成为隐含动量与能量的来源。
以界面带守恒收支误差、静力平衡偏差与非物理流动强度等作为可检验判据,
为相变更新提供可靠界面条件。

\ct{(2)}\hl{同步节点组织下的多时间尺度推进与拓扑更新约束}
	
相变诱导体积剧变会导致粒子尺度变化与邻域退化,并与两相稳定步长差异叠加,
使多时间尺度推进与拓扑更新不可避免。若界面交换、相变更新与拓扑操作发生
在不同时间层,误差易跨步累积并放大,表现为压力噪声增强、相变速率波动与
守恒误差增长。
本项目的关键技术是以\ct{同步节点}统一组织跨相更新:在同步节点集中完成
界面交换与相变更新,拓扑更新限定在同步节点附近执行,并在\ct{守恒约束}
下完成变量重分配,控制拓扑变化对邻域一致性与界面几何量的扰动。
以跨步守恒误差增长率、压力噪声水平与相变速率波动幅值作为核心判据,保证
体积剧变场景下的稳定计算范围。

\section{本研究特色创新之处}

本项目面向低温相变换热等重大工程场景的高保真数值模拟需求,聚焦
强物性跃变与相变诱导体积剧变条件下,分离式两相SPH易出现界面压
力速度传递不稳、相变能量误差累积与多时间尺度错配等问题,开展守
恒一致性耦合算法研究。
项目以\ct{界面动量一致交换—相变潜热守恒更新—同步守恒推进}为
主线,构建统一离散与时间层组织框架,形成可复用的算法模块与验证
算例体系,为极端工况下相变两相流模拟提供可验证的方法支撑。

\hl{本项目的特色}在于将界面耦合、相变更新与体积剧变处理置于同
一求解流程内协同设计,而非将三者作为相互独立的补丁模块串联。
具体而言:界面动量交换以守恒型成对通量进入两相推进,并与压力边
界与速度约束的施加方式一致匹配;
相变质量通量由界面热通量闭合并与潜热收支同步守恒更新,再以同一
时间层源项一致嵌入两相求解;
拓扑更新仅在同步节点处执行,并在守恒约束下完成变量重分配,以避
免重构破坏界面交换与相变更新的一致性。
上述组织使稳定性、守恒性与时序一致性在同一离散体系内同时受控,
提升了模型在极端工况下的可复现性与可扩展性。

\hl{本项目的创新点}体现在对三类关键环节给出可检验的一致性构造与
约束条件:
\textcircled{\raisebox{-0.8pt}{1}}从依赖平滑扩散抑制转向压力
速度传递的离散相容约束,以界面接触态驱动的守恒型通量统一压力信息
与法向速度条件,降低界面误差反复注入导致的非物理流动与压力振荡;
\textcircled{\raisebox{-0.8pt}{2}} 从相变作为外加源项转向热通
量驱动的质量通量闭合,在离散层面同步更新质量交换与潜热收支,并与
两相求解一致嵌入,降低温度过冲、假相变与能量漂移;
\textcircled{\raisebox{-0.8pt}{3}} 从仅维持分辨率的重构转向受
时序与守恒性双重约束的拓扑更新,以同步节点组织多速率推进,并在守
恒约束下最小扰动重分配,使分裂合并与多时间尺度对齐,抑制跨步误差
累积。
上述创新将方法可靠性由经验调参推进到离散结构一致可验证,并形成可
推广的算法模块与检验规范。
%
\section{年度研究计划及预期研究结果}
\subsection{年度研究计划}
本项目执行期限为3年(2027年1月至2029年12月),年度计划如下:

(1)2027年1月$\sim$ 2027年12月

完成强跃变界面动量一致交换的离散表述,形成界面压力边界与速度约
束的一致施加方法与邻域退化条件下界面力稳定化方法;
构建非相变两相基准算例,完成稳定性、守恒一致性与收敛性对比验证;
拟参加学术会议1次,发表高质量学术论文1篇。

(2)2028年1月$\sim$ 2028年12月

完成界面热通量的稳定离散与计算,建立热通量驱动的相变质量通量闭合
与潜热守恒更新方法;
明确相变质量与能量更新进入两相求解的同时间层嵌入方式,实现与界面
交换的协调推进,并在相变两相算例中开展质量与能量守恒与稳定性验证。
拟参加学术会议1-2次,发表高质量学术论文1-2篇。

(3)2029年1月$\sim$ 2029年12月

建立多时间尺度推进的跨相同步更新规则,提出分裂合并与变量重分配的
守恒约束并明确同步节点处嵌入位置;
完成三模块一体化集成,在强耦合综合算例中开展系统验证与定量对比评估;
完成项目总结与结题报告撰写,凝练可复用算法模块与验证算例;
拟申请软件著作权1-2项,参加学术会议1-2次,发表高质量学术论文1-2篇,
培养研究生1名。

%
\subsection{预期研究成果}
通过本项目的研究,预期取得以下成果:
\begin{itemize}[left= 25pt]
	\item[$\bullet$] 构建界面动量一致交换与稳定耦合方法,形成可复用代码与验证算例;
	\item[$\bullet$] 建立热通量驱动的相变闭合与潜热守恒更新流程,完成守恒性验证;
	\item[$\bullet$] 提出多时间尺度同步推进与守恒约束拓扑更新方法,完成集成评估;
	\item[$\bullet$] 在国内外重要学术期刊或会议上发表学术论文3-5篇;
	\item[$\bullet$] 形成可复用程序原型与算法模块1套(可登记软件著作权1-2项);
	\item[$\bullet$] 参加国内或国际会议3-5次;
	\item[$\bullet$] 培养硕士研究生1名。
\end{itemize}
%
\newpage