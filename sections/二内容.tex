%
\section{研究目标与关键科学问题}
\subsection{总体研究目标}
本项目面向低温相变换热等典型工况中的气液相变两相流数值模拟需求,
发展基于无网格SPH的守恒一致耦合方法。
针对强物性差异与相变诱导体积剧变条件下的计算难题,需要同时应对
“界面稳定耦合”“相变守恒更新”与“粒子守恒重构”三方面挑战,着重
解决\textbf{\textcolor{B}{“相界面物性
		跃变处压力-速度信息的稳定传递与界面耦合一致离散”}}和
\textbf{\textcolor{B}{“相变闭合与粒子自适应在统一离散框架
		下的守恒一致实现”}}两大
核心科学问题。总体研究思路如图\ref{总体研究思路}所示。

%
\begin{figure}[h!]
	\centering %图片居中
	\includegraphics[width=14.6cm]{figures/总体研究思路}
	\captionsetup{justification=centering} %图题居中
	\caption{本项目总体研究思路示意图}\label{总体研究思路}
\end{figure}
%

本项目属于无网格法(SPH)数值算法的基础研究,主要研究目标包括:

(1)\textbf{\textcolor{B}{界面耦合}}:
建立相界面物性跃变处压力–速度信息的稳定传递方法,完善界面边界
条件与约束的施加与交换机制,使两相耦合推进在高密度比条件下抑制
界面压力振荡与寄生流。

(2)\textbf{\textcolor{B}{相变更新}}:
建立热通量驱动的相变质量通量闭合与潜热交换守恒更新方法,提出与
两相耦合推进相协调的相变更新策略,降低温度过冲、假相变与能量
漂移,提升相变速率计算的稳定性与可控性。

(3)\textbf{\textcolor{B}{体积剧变}}:建立体积剧变条件下
带守恒约束的粒子分裂/合并与变量重分配策略,并与多时间尺度推进
相协同,保证粒子自适应重构不破坏界面耦合与相变闭合的一致性,
维持计算精度与稳定性。

在上述目标基础上,形成一套\textbf{\textcolor{B}{面向气液相变
两相流的分离式SPH守恒一致耦合求解方法}},实现界面耦合、相变闭
合与粒子自适应在统一离散框架下的稳定协同,并通过由简到繁的
典型算例开展系统检验。

\subsection{拟解决的关键科学问题}
本项目拟解决的关键科学问题如下:

(1)\textbf{\textcolor{B}{相界面物性跃变处压力-速度信息
		的稳定传递与界面耦合一致离散}}。
分离式两相SPH中,液相与气相推进策略不同,界面处需通过压力
边界与速度约束实现双向耦合。
在高密度比与邻域退化条件下,核支撑域截断与粒子扰动易放大界面
离散误差,诱发压力振荡、寄生流与相间掺混,并导致稳定域收缩。
\textbf{\textcolor{B}{关键在于界面边界条件的离散构造与
		交换方式未与两相推进过程一致匹配}},界面误差易进入压
力-速度更新并被持续放大。
需要在受限核支撑与界面几何约束下,统一压力边界重构、速度约束
施加与稳定化强度选取,使界面传递与推进算子保持同构一致。
因此,本项目将构建与推进算子一致的界面边界离散与交换机制,
形成可控的压力-速度稳定传递策略,为高密度比两相计算提供稳健
界面耦合框架。		

(2)\textbf{\textcolor{B}{相变闭合与粒子自适应在统一
		离散框架下的守恒一致实现}}。
气液相变由界面热通量驱动,质量交换与潜热更新会反馈影响压力
求解与速度更新;相变诱导的体积膨胀又会导致邻域退化并触发
粒子重构需求。
相变闭合、潜热守恒更新与粒子分裂/合并若分步实施,守恒误差
与时间错配易累积,表现为能量漂移、潜热重复计入或相变速率波
动,并削弱整体稳定性。
\textbf{\textcolor{B}{关键在于相变更新与粒子拓扑重构未
		纳入统一守恒更新流程并实现时间对齐}},局部误差会
反向扰动界面耦合稳定性。
需要将热通量驱动的相变闭合、潜热守恒更新、分裂/合并后的
变量重分配与多时间尺度界面交换纳入同一更新链路,避免守恒
与时序错配的累积。
因此,本项目将构建相变闭合与潜热守恒更新方法,并与粒子自
适应重构及多时间尺度推进协同设计,实现体积剧变条件下相变
两相流的稳定可靠计算。				
%
\section{研究内容}
为达到上述目标,本项目拟开展三个方面的研究工作。 研究内容与
相互支撑关系如图\ref{研究内容}所示。具体研究内容阐述如下:

\begin{figure}[h!]
	\centering %图片居中
	\includegraphics[width=14.6cm]{figures/研究内容}
	\captionsetup{justification=centering} %图题居中
	\caption{本项目拟开展的研究内容}\label{研究内容}
\end{figure}

\textbf{\textcolor{B}{研究内容1:相界面压力-速度信息稳定
		传递的界面耦合离散与推进方法}}
\begin{enumerate}
	\item[] \textbf{(1)界面边界条件的离散构造与交换形式统一:}
	面向分离式两相SPH求解流程,针对两相可压性描述与推进方式差异,
	在受限核支撑与界面几何约束下统一压力边界重构与速度约束施加的
	离散表达,使两者共享一致的界面几何权重、法向信息与邻域支撑数据;
	同时明确界面信息在各变量更新链路中的调用与交换顺序,避免同一
	时间层中界面输入不一致导致的误差注入与耦合漂移。
	\item[] \textbf{(2)邻域退化下的误差抑制与界面力平衡稳定化:}
	针对核支撑域截断与粒子扰动引起的边界误差放大,研究与界面离散
	形式相协调的稳定化处理与强度选取,使稳定化在抑制压力振荡的
	同时不破坏守恒一致性。进一步将表面张力等界面力纳入界面耦合
	离散的一致框架,统一界面法向/曲率评估与梯度/散度修正,保证
	界面力与压力–速度传递同构匹配,抑制寄生流与相间掺混的放大。
	\item[] \textbf{(3)耦合推进流程的同构组织与接口定义:}
	围绕“界面交换—压力求解—速度更新”的时间推进链路,研究在同一
	时间层内的同构组织方式,明确界面交换量、更新顺序与必要的修正
	环节,形成可与相变更新与粒子自适应模块直接对接的界面推进框架,
	并在典型两相算例中检验其稳定性与守恒一致性表现。
\end{enumerate}

\textbf{\textcolor{B}{研究内容2:热通量驱动的相变闭合离散与潜热
		守恒更新方法}}
\begin{enumerate}
	\item[] \textbf{(1)物性跃变条件下界面热通量的稳定评估与离散表达:}
	针对强温差与物性跃变导致的界面温度梯度估计不稳问题,研究能量方程
	在SPH框架下的离散计算方式,构建与界面几何一致的热通量评估方法,
	使两侧热通量与离散算子在同一权重体系下自洽;
	在不依赖平滑的前提下提高梯度估计稳定性,降低温度场数值振荡风险,
	为相变质量通量计算提供可靠输入。
	\item[] \textbf{(2)相变质量通量闭合与潜热守恒更新链路:}
	基于界面能量平衡关系,建立由界面热通量驱动的相变质量通量计算
	与闭合表达,在离散层面实现质量交换与潜热释放/吸收的守恒更新,
	抑制温度过冲、假相变与能量漂移;
	同时明确相变更新对局部密度、体积与状态量的更新顺序与耦合变量
	一致性要求,避免潜热重复计入与收支失衡的累积。
	\item[] \textbf{(3)相变更新与两相耦合推进的时间层协调:}
	研究相变质量/能量更新与压力求解、速度更新之间的耦合组织方式,
	将相变更新纳入统一时间层推进流程,给出与两相耦合推进相协调的
	更新顺序与源项处理策略,降低相变对压力-速度更新的扰动放大;
	通过典型相变算例对温度场、相变速率与能量收支进行一致性检验,
	形成可集成的相变闭合与守恒更新模块。
\end{enumerate}

\textbf{\textcolor{B}{研究内容3:体积剧变下粒子自适应重构的守恒
		实现与多时间尺度推进}}
\begin{enumerate}
	\item[] \textbf{(1)体积剧变区域的粒子自适应触发与一致重构流程:}
	面向相变诱导体积膨胀导致的粒子稀疏与邻域退化,建立分裂/合并触
	发指标与重构流程,使粒子分辨率与邻域支撑在体积剧变过程中保持可控;
	重点约束相界面附近与相变活跃区的粒子表示质量,避免邻域退化引起
	界面通量计算失真并破坏相变闭合。
	\item[] \textbf{(2)分裂/合并后的守恒变量重分配与一致修正:}
	构建分裂/合并与重采样后的质量、动量与能量变量重分配方法,保证
	局部守恒并与界面耦合与相变闭合的离散形式一致;
	通过一致重映射与必要的修正环节,减少重构引入的隐性质量/能量
	误差与压力扰动,避免潜热重复计量在重构过程中累积。
	\item[] \textbf{(3)多时间尺度推进的组织与界面交换对齐:}
	针对声学CFL、热扩散与相变更新带来的多时间尺度约束,研究分层时间
	推进组织方式,建立气相子步与液相大步耦合推进的界面交换对齐策略,
	明确界面交换、相变更新与粒子重构的时间层协同安排,避免时序错配
	导致的相变速率波动与界面耦合不稳;并在体积剧变相变算例中检验
	推进效率与稳定裕度。
\end{enumerate}

\textbf{\textcolor{B}{研究内容间支撑关系:}}
三项研究以守恒一致更新流程为主线,两两耦合、相互约束,其关系如图
\ref{研究内容}所示。
研究内容1提供稳定的界面压力-速度传递与几何输入,支撑研究内容2的
热通量评估与相变闭合。
相变引起的质量/能量变化与体积膨胀将反向影响界面传递精度并触发粒子
自适应需求,研究内容3通过守恒重分配与时间层对齐,约束并保障研究内
容1与2在体积剧变条件下的稳定与一致。
%		
\section{拟采取的研究方案}

\subsection{技术路线}

\begin{figure}[h!]
	\centering %图片居中
	\includegraphics[width=14.6cm]{figures/技术路线}
	\captionsetup{justification=centering} %图题居中
	\caption{本项目拟采用的技术路线}\label{技术路线}
\end{figure}

本项目研究将主要遵循图\ref{技术路线}所示的技术路线,面向强
物性跃变与相变诱导体积剧变条件下气液相变两相流的高保真数值
模拟需求,以分离式SPH守恒一致性耦合框架为主线,综合采用离散
算子构造、耦合推进组织与典型算例定量验证相结合的方式开展研究;
总体上遵循“界面压力速度稳定耦合离散与推进→热通量驱动相变闭合
与潜热守恒更新→体积剧变粒子自适应重构与多时间尺度对齐集成验证”
的研究路线。

首先,从界面耦合稳定性入手,统一压力边界与速度约束的离散表达,
针对邻域退化建立误差抑制和稳定耦合策略,形成“界面交换-压力求解-
速度更新”的一致推进方案,并在两相静水保持、拉普拉斯压差、气泡
上升及剪切层不稳定等非相变基准算例逐层开展稳定性与一致性验证;
其次,面向流-热-相变强耦合过程,构建界面热通量稳定评估与离散
方法,建立热通量驱动的相变质量通量闭合与潜热守恒更新链路,并与
一致推进方案实现时序对齐,通过K跃变的两材料导热、高热六密度导热
测试、1D/2D Stefan问题、平面界面蒸发/冷凝及受热壁气泡生长以及
冷凝气泡收缩等算例进行逐层定量验证与适用范围评估。
最后提出体积剧变触发的粒子自适应重构准则,建立分裂/合并后的守恒
量重分配与误差控制方法,形成多时间尺度推进与界面交换对齐策略,
并在快速膨胀气泡、局部相变稀疏测试、分裂/合并回归、气液两相耦合
及综合相变两相算例中完成集成验证与参数建议凝练。

通过上述递进式研究,最终形成一套面向气液相变两相流的SPH守恒一致
性耦合求解方法,实现界面耦合、相变闭合与粒子自适应在统一离散框
架下的稳定协同,并为相关极端工况的高置信度数值预测提供可复用的
算法模块。

\subsection{研究方法}
结合技术路线图,详细阐述具体的研究方法如下:

(1)\textbf{\textcolor{B}{界面压力-速度稳定耦合离散与推进方法
		(针对研究内容1)}}

现有分离式气液两相粒子法中,典型思路是采用气相可压显式(WCSPH)与
液相不可压投影(ISPH)以避免时间步长被气相声速锁死,并通过界面边界
条件完成两相耦合,即WCSPH侧提供界面压力边界、ISPH侧提供界面速度
边界。
本项目在此基础上,面向高密度比与界面邻域退化,即核支撑域截断条件,
将压力边界重构-泊松求解-速度约束施加组织为同一时间层的闭环更新,
并以算子相容性约束实现动量/体积守恒的一致性耦合。

液相不可压推进采用投影框架,先以显式形式得到预测速度
\begin{equation}
	u^*=u^n+\frac{\Delta t(F_v + F_s + F_g)}{\rho},
\end{equation}
再通过压力泊松方程(PPE)求得$p^{n+1}$并修正速度
\begin{equation}
	u^{n+1} = u* − \frac{\Delta t(\nabla p^{n+1})}{\rho}.
\end{equation}
PPE采用变系数形式
\begin{equation}
	\nabla\cdot(\frac{1}{\rho}\nabla p^{n+1})=\frac{1}{\Delta t}
	\nabla\cdot u^*.
\end{equation}
由此可直接给出不可压性的数理保证:对速度修正式取散度,得
\begin{equation}
	\nabla\cdot u^{n+1}=\nabla\cdot u^*-\Delta t\nabla\cdot(
	\frac{1}{\rho}\nabla p^{n+1})=0,
\end{equation}
最后一步由PPE右端定义成立,因此离散层面PPE—速度修正构成严格的散度
约束闭环,误差来源仅来自PPE离散/迭代残差。
该闭环的关键在于界面压力边界的构造:已有混合框架通常通过界面插值/平
均将气相压力施加为液相PPE的狄里赫雷边界或等效边界源项,并强调由此实
现应力平衡与稳定耦合。
为降低支撑域截断导致的压力边界噪声,本项目拟采用\textbf{\textcolor{B}
{相容加权重构的界面压力}},即以气相邻域压力的核加权平均作为边界压力
\begin{equation}
	P_i=\frac{\sum_{j\in air} P_j W_{ij}}{\sum_{j\in air} W_{ij}},
\end{equation}
并在压力梯度离散中保持对称/虚功一致结构,即基于虚功原理的压力梯度离
散为
\begin{equation}
	\nabla p_i = \frac{1}{V_i}\sum_j(p_i V_i^2+p_j V_j^2)\nabla_i 
	W_ij, 
\end{equation}
可保证成对作用力反对称从而维持动量守恒。

针对跨相邻域的压力不连续/邻域退化问题,已有CISPH耦合中引入跨相修正
压力$p^*_j$以增强界面压力连续性与稳定性;
本项目拟将该类修正与界面压力重构统一到同一离散算子族中,使压力边界-
梯度算子-PPE共享同一核函数一致性约束,从源头减少非物理压力振荡与流动。
气相速度边界则由液相界面或边界粒子速度提供,将液相界面速度作为气相的
动边界速度条件以位处界面速度连续,并与液相修正速度$u^{n+1}$同步更新,
避免跨时间层传递造成的相位误差累积。
最终以界面寄生流强度、静水平衡压力误差、R-P气泡振荡压力与半径时程等
为定量指标开展验证。

(2)\textbf{\textcolor{B}{热通量驱动相变闭合与潜热守恒更新方法
		(针对研究内容2)}}
	
相变两相SPH的核心难点在于温度场/物性,如导热系数、热扩散率等在界面
处可能强间断,若仍沿用常规扩散离散,易出现界面附近温度低估或振荡不
收敛。
已有研究多通过修正跨相等效导热系数来减弱初始间断误差,例如取
\begin{equation}
	\lambda_{ab}=\frac{4\lambda_{a}\lambda_{b}}{(\lambda_{a}+
		\lambda_{b})},
\end{equation}
并配套给出热传导的SPH离散
\begin{equation}
	\frac{dT_a}{dt}=\sum_b \frac{2m_b \lambda_{ab}(T_{a}-T_{b})}
	{\rho_a \rho_b(r_{ab}^2+\eta^2)}r_{ab}\nabla W_{ab}.
\end{equation}
本项目拟在上述等效导热系数与对称扩散离散的基础上,进一步引入Riemann
近似思想处理温度间断,将界面接触温度取为$T^*_{ij}=(T_i+T_j)/2$,
并据此重写扩散项,使温度跳跃被结构性地对半分配到两侧粒子更新中。
对应的Riemann-SPH导热模型可写为
\begin{equation}
	\frac{dT_i}{dt}=\frac{1}{\rho_i}\sum_j\frac{4 k_i k_j}{(k_i+k_j)}
	\cdot2 (T_i-T^*_{ij})\cdot\frac{r_{ij}}{(r_{ij}^2+\eta^2)}
	\nabla_i W_ij\cdot V_j.
\end{equation}
进一步地,为适配高热扩散率比导致的瞬态热边界层厚度差异,可将热扩散率比
$\alpha_i/\alpha_j$显式并入权重,得到考虑热扩散率比的Riemann-SPH形式。
该路线的优势在文献中已通过1D间断温度导热与二维腔体导热的精度/收敛性对比
得到验证。

在相变闭合方面,本项目以热通量驱动的界面质通量构建质量与能量方程的守恒
耦合。已有相变粒子法常将相变质量源项并入连续方程与能量方程
\begin{equation}
	\displaystyle\dfrac{D\rho}{Dt}=−\rho\nabla\cdot u+\dot{m},
\end{equation}
$(\nabla\cdot(\lambda \nabla T)-h_{lv} \dot{m})/(\rho C_p)$形式进入
温度演化。
由此可得到清晰的能量守恒解释,对单粒子控制体积$V$,有$\rho C_p V (dT/dt)$
= (导热净输入) − $h_{lv}(\dot{m}V)$,即相变消耗的潜热由界面导热供给,或
相反,从而保证热—质闭合不引入非物理能量源。
为避免数值超调,本项目拟采用饱和温度约束与相变通量限幅结合的策略,即仅当
界面附近满足$T\geq T_{sat}$(汽化)或 $T\leq T_{sat}$(凝结)时激活
$\dot{m}$,并将$\dot{m}$与离散导热通量一致地绑定到同一核支撑域统计量上,
使$\dot{m}$对粒子扰动的敏感性被扩散算子的对称结构抑制。
相变基准验证采用Stefan问题(界面位置解析解)、汽泡凝结/蒸发体积演化等
文献算例体系。

(3)\textbf{\textcolor{B}{体积剧变粒子自适应重构与多时间尺度对齐方法
		(针对研究内容3)}}
	
相变导致的体积膨胀/收缩会快速拉大粒子尺度差异并诱发邻域退化,从而破坏
前述界面耦合与传热离散的稳定域。
已有相变粒子法通常将相变质量更新—粒子体积更新—粒子分裂/合并作为独立模块
串联,并指出为避免粒子尺度差异必须引入分裂/合并机制;
例如当气泡粒子体积超过阈值时触发分裂,并在实现上以质量/密度/体积更新后
进入split-merge模块,再进入运动求解的流水线组织。
本项目拟将该类拓扑重构提升为守恒约束下的最小扰动重构:对分裂(或合并)
操作显式施加三类守恒条件
\begin{equation}
	\begin{cases}
		\sum m_k = m_{parent}\\
		\sum m_k u_k = m_{parent} u_{parent}\\
		\sum m_k e_k = m_{parent} e_{parent}
	\end{cases}
\end{equation}
其中$e_k$为单位质量总能(含显热与潜热份额),保证拓扑重构不会成为隐含的
动量/能量源;
同时通过局部重心保持与核一致重采样,使新粒子对邻域核矩的一阶一致性扰动
最小,从而与\textbf{\textcolor{B}{研究内容一的算子相容性要求闭合}}。

此外,分离式气液两相推进天然存在多时间尺度:气相显式步长与液相PPE步长通常不同。
已有耦合框架常将两相时间步设置为整数倍并做子循环,以避免界面条件在非对齐时
间节点上传递而产生相位误差;
例如相变两相耦合中明确提出将时间步比设为整数倍以同步推进,而混合CISPH框架中也
以满足某一模运算条件时执行ISPH步的方式实现子循环对齐。
本项目拟进一步将“时间对齐”写成硬约束:设$\Delta t_L$为液相步长、$\Delta t_G$
为气相步长,取$N=\Delta t_L/\Delta t_G\in N$,并在每个液相步内进行$N$次气
相子步;
界面量($P_i$、$u_i$、$T_i$、$\dot{m}$)仅在$t^n$与$t^{n+1}$两个对齐节点
完成一次一致交换(必要时在子步内使用守恒插值预测),从结构上消除界面条件时序
漂移。
在此对齐框架中,拓扑重构(split-merge)被限定只在对齐节点执行,并与质量/能量
方程同步更新,从而保证相变引起的体积剧变只通过$\dot{m}$—$V$—粒子拓扑这一
条守恒链路进入动量与压力求解,避免在子步中间插入结构突变导致的压力泊松求解
不收敛或寄生流突增。

最终以“强相变体积剧变(汽泡快速生长/塌缩)与高密度比界面耦合综合算例,定量
评估拓扑重构与时间对齐对稳定域与计算效率的提升,并以收敛性($\Delta x$→0)、
守恒误差(质量/动量/能量)与寄生流强度作为核心指标。
\subsection{关键技术}

(1)\textbf{\textcolor{B}{高密度比条件下界面压–速边界的一致构造
		与闭环推进}}
	
分离式气液两相 SPH 中,界面耦合通常通过气相提供压力边界、液相施加速度
约束实现。
高密度比与邻域退化条件下,核支撑域截断与粒子扰动使界面边界误差被持续注入
压力求解与速度更新过程,易导致压力振荡、寄生流与相间掺混,耦合稳定域显著
收缩。
本项目拟围绕界面交换—压力求解—速度更新的同一时间层推进链路,统一压力边界
重构与速度约束施加的离散表达与交换顺序,使两类边界条件共享一致的界面几何
信息与邻域权重,并与压力求解/速度修正保持离散结构相容,从而形成可控的界
面压–速闭环耦合推进框架,为后续相变更新与粒子自适应模块提供稳定接口。	

(2)\textbf{\textcolor{B}{热通量驱动相变守恒更新与体积剧变粒子重构的
		协同约束}}
	
相变过程由界面热通量驱动,质量交换与潜热释放/吸收会改变局部密度与体积分布
并反馈影响压–速求解;相变诱导体积剧变又会造成粒子稀疏与邻域退化,触发分
裂/合并等拓扑调整。
若相变闭合、潜热更新与粒子重构按模块分步叠加,守恒误差与时间错配易累积,
表现为温度过冲、假相变、能量漂移与相变速率波动,并反向削弱界面耦合稳定性。
本项目拟构建与界面几何一致的热通量稳定评估与相变质量通量闭合,在离散层面
实现潜热交换的守恒更新;同时将粒子分裂/合并后的守恒量重分配、多时间尺度子
步推进与界面交换对齐纳入统一更新流程,确保体积剧变条件下相变闭合、界面耦合
与粒子自适应之间的协同一致。	

\section{本研究特色创新之处}

本项目面向低温相变换热等重大工程应用中的高可信数值模拟需求,针对
强物性差异与相变诱导体积剧变条件下分离式两相SPH易出现压力速度耦合
不稳、守恒误差累计与时序错配等问题,开展守恒一致性耦合算法研究。
项目将构建“界面压力速度稳定耦合-热通量驱动相变守恒更新-体积剧变
粒子守恒重构与多时间尺度对齐”的统一离散与时间层组织框架,形成可
复现、可验证的模块化求解框架。
预期可提升相变两相流SPH模拟的稳定性与守恒一致性,为低温相变换热
过程的数值分析与工程设计提供可靠的方法支撑。

\textbf{本项目的特色}在于以“守恒一致性更新流程”为主线,面向强物性
跃变与相变诱导体积剧变工况,从方法论层面把界面耦合、相变更新与粒子
自适应视为同一离散体系下的耦合约束。
相较现有研究采用“模块化叠加”与“状态量交换”的弱耦合路线,本项目强调
三者在离散层面相互反馈:界面离散会被推进链路放大,相变源项会改变
压力约束与体积分布,拓扑重构又会改变邻域与几何输入,因此必须在统一
口径下协同设计。
本项目将通过统一的数据接口与时间层组织,把稳定性、守恒性与时序一致
性作为同等约束同时落实,从而提升极端工况下SPH计算的可靠性与拓展性。

\textbf{本项目的创新之处}在于面向极端耦合工况,将“界面耦合-相变
更新-粒子重构”由经验修补的分布策略,提升为可检验的离散一致性构造。
具体表现为:\textcircled{\raisebox{-0.8pt}{1}}由平滑/阻尼抑制
转向压速传递离散相容性约束,使界面稳定性建立在离散结构匹配而非参数
敏感性上;\textcircled{\raisebox{-0.8pt}{2}}由相变作为外加源项
转向相变纳入守恒更新链路,在离散层面闭合质量交换与潜热收支,降低
能量漂移与假相变;\textcircled{\raisebox{-0.8pt}{3}}由重构只保
分辨率转向重构受守恒与时序双约束,将分裂/合并与多时间尺度推进对齐,
避免拓扑操作反向破坏界面与相变计算。
上述创新把方法的可靠性从调参可用推进到结构一致可复现,形成可推广
的算法范式与通用模块。
%
\section{年度研究计划及预期研究结果}
\subsection{年度研究计划}
本项目执行期限为三年(2027年1月至2029年12月),年度研究计划如表1
所示。

\begin{figure}[h!]
	\centering %图片居中
	\includegraphics[width=14.7cm]{figures/年度计划}
	\captionsetup{justification=centering} %图题居中
	\caption{本项目研究计划安排}
\end{figure}

具体安排如下:

\textbf{}(1)2027年1月$\sim$ 2027年12月

完成界面压力边界与速度约束的统一离散,构建邻域退化条件下界面力稳定化
耦合策略,形成“界面交换—压力求解—速度更新”的一致推进框架并实现误差可控;
构建非相变两相基准算例,完成稳定性、守恒一致性与收敛性对比验证;
拟参加学术会议1次,发表高质量学术论文1篇。

\textbf{}(2)2028年1月$\sim$ 2028年12月

完成界面热通量的稳定评估与离散,建立热通量驱动的相变闭合与潜热守恒更新
方法;
明确相变更新与界面推进的时序协同与必要修正环节,实现相变模块与界面推进
框架的时间对齐与稳定耦合;
在相变两相算例中开展质量/能量守恒与稳定性验证,给出相变闭合方法的适用
范围与参数建议;
拟参加学术会议1-2次,发表高质量学术论文1-2篇。

\textbf{}(3)2029年1月$\sim$ 2029年12月

建立体积剧变触发的粒子自适应重构准则,完成分裂/合并后的守恒量分配与
修正方法,
形成多时间尺度推进与界面交换对齐策略;
完成三模块一体化集成,在综合算例中开展系统验证与定量对比评估;
完成项目总结与结题报告撰写,凝练可复用算法模块与推荐参数区间;
拟申请软件著作权1-2项,参加学术会议1-2次,发表高质量学术论文1-2篇,
培养研究生1名。

%
\subsection{预期研究成果}
通过本项目的研究,预期取得以下成果:
\begin{itemize}[left= 25pt]
	\item[$\bullet$] 构建界面压力速度稳定耦合离散方法,形成可复现代码与基准算例集;
	\item[$\bullet$] 建立热通量驱动的相变闭合与潜热守恒更新流程,完成守恒性验证;
	\item[$\bullet$] 提出体积剧变下粒子守恒重构与多时间尺度对齐策略,完成集成评估;
	\item[$\bullet$] 在国内外重要学术期刊或会议上发表学术论文3-5篇;
	\item[$\bullet$] 形成可复用程序原型与算法模块1套(可登记软件著作权1-2项);
	\item[$\bullet$] 参加国内或国际会议3-5次;
	\item[$\bullet$] 培养硕士研究生1名。
\end{itemize}
%
\newpage