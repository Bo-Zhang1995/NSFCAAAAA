%
\section{研究目标}
本项目面向低温相变换热等典型工况中的气液相变两相流数值模拟需求,
发展基于无网格SPH的守恒一致耦合方法。
针对强物性差异与相变诱导体积剧变条件下的计算难题,需要同时应对
“界面两相稳定耦合”“相变潜热守恒更新”与“粒子自适应重构的可控
守恒”三方面挑战,着重解决\textbf{\textcolor{B}{“相界面物性
		跃变处压力-速度信息的稳定传递与界面耦合一致离散”}}和
\textbf{\textcolor{B}{“相变闭合与粒子自适应在统一离散框架
		下的守恒一致实现”}}两大
核心科学问题。总体研究思路如图\ref{总体研究思路}所示。

%
\begin{figure}[h!]
	\centering %图片居中
	\includegraphics[width=5cm]{figures/xiugai.png}
	\captionsetup{justification=centering} %图题居中
	\caption{本项目总体研究思路示意图}\label{总体研究思路}
\end{figure}
%

本项目属于无网格法(SPH)数值算法的基础研究,主要研究目标包括:

(1)\textbf{\textcolor{B}{界面耦合}}:
建立相界面物性跃变处压力–速度信息的稳定传递方法,完善界面边界
条件与约束的施加与交换机制,使两相耦合推进在高密度比条件下抑制
界面压力振荡与寄生流。

(2)\textbf{\textcolor{B}{相变更新}}:
建立热通量驱动的相变质量通量闭合与潜热交换守恒更新方法,提出与
两相耦合推进相协调的相变更新策略,降低温度过冲、假相变与能量
漂移,提升相变速率计算的稳定性与可控性。

(3)\textbf{\textcolor{B}{体积剧变}}:建立体积剧变条件下
带守恒约束的粒子分裂/合并与变量重分配策略,并与多时间尺度推进
相协同,保证粒子自适应重构不破坏界面耦合与相变闭合的一致性,
维持计算精度与稳定性。

在上述目标基础上,形成一套\textbf{\textcolor{B}{面向气液相变
		两相流的分离式SPH守恒一致耦合求解方法}},实现界面耦合、相变闭
合与粒子自适应在统一离散框架下的稳定协同,并通过由简到繁的
典型算例开展系统检验。
%
\section{研究内容}
为达到上述目标,本项目拟开展三个方面的研究工作。 研究内容与
相互支撑关系如图\ref{研究内容}所示。具体研究内容阐述如下:
%
\begin{figure}[h!]
	\centering %图片居中
	\includegraphics[width=5.5cm]{figures/xiugai.png}
	\captionsetup{justification=centering} %图题居中
	\caption{本项目拟开展的研究内容}\label{研究内容}
\end{figure}
%

\textbf{\textcolor{B}{研究内容1:相界面压力-速度信息稳定
		传递的界面耦合离散与推进方法}}
\begin{enumerate}
	\item[] \textbf{(1)界面边界条件的离散构造与交换形式统一:}
	面向分离式两相SPH求解流程,针对两相可压性描述与推进方式差异,
	在受限核支撑与界面几何约束下统一压力边界重构与速度约束施加的
	离散表达,使两者共享一致的界面几何权重、法向信息与邻域支撑数据;
	同时明确界面信息在各变量更新链路中的调用与交换顺序,避免同一
	时间层中界面输入不一致导致的误差注入与耦合漂移。
	\item[] \textbf{(2)邻域退化条件下的界面误差抑制与稳定化强度选取:}
	针对核支撑域截断与粒子扰动引起的边界误差放大,研究与界面离散
	形式相协调的稳定化处理,并给出稳定化强度的可控选取方式,使振荡
	抑制不以破坏守恒一致为代价;
	结合界面几何表征,构建界面附近梯度/散度的一致修正与耦合算子相容
	化处理,抑制界面压力振荡向速度场回灌并抑制寄生流放大,提高高密度
	比条件下的稳健性。
	\item[] \textbf{(3)耦合推进流程的同构组织与接口定义:}
	围绕“界面交换—压力求解—速度更新”的时间推进链路,研究在同一时间
	层内的同构组织方式,明确界面交换量、更新顺序与必要的修正环节,
	形成可与相变更新与粒子自适应模块直接对接的界面推进框架,并在典型
	两相算例中检验其稳定性与守恒一致性表现。
\end{enumerate}

\textbf{\textcolor{B}{研究内容2:热通量驱动的相变闭合离散与潜热
		守恒更新方法}}
\begin{enumerate}
	\item[] \textbf{(1)物性跃变条件下界面热通量的稳定评估与离散表达:}
	针对强温差与物性跃变导致的界面温度梯度估计不稳问题,研究能量方程
	在SPH框架下的离散计算方式,构建与界面几何一致的热通量评估方法,
	使两侧热通量与离散算子在同一权重体系下自洽;
	在不依赖平滑的前提下提高梯度估计稳定性,降低温度场数值振荡风险,
	为相变质量通量计算提供可靠输入。
	\item[] \textbf{(2)相变质量通量闭合与潜热守恒更新链路:}
	基于界面能量平衡关系,建立由界面热通量驱动的相变质量通量计算与
	闭合表达,在离散层面实现质量交换与潜热释放/吸收的守恒更新,抑制
	温度过冲、假相变与能量漂移;
	同时明确相变更新对局部密度、体积与状态量的更新顺序与耦合变量一致
	性要求,避免潜热重复计入与收支失衡的累积。
	\item[] \textbf{(3)相变更新与两相耦合推进的时间层协调:}
	研究相变质量/能量更新与压力求解、速度更新之间的耦合组织方式,将
	相变更新纳入统一时间层推进流程,给出与两相耦合推进相协调的更新顺序
	与源项处理策略,降低相变对压力-速度更新的扰动放大;
	通过典型相变算例对温度场、相变速率与能量收支进行一致性检验,形成
	可集成的相变闭合与守恒更新模块。
\end{enumerate}

\textbf{\textcolor{B}{研究内容3:体积剧变下粒子自适应重构的守恒
		实现与多时间尺度推进}}
\begin{enumerate}
	\item[] \textbf{(1)体积剧变区域的粒子自适应触发与一致重构流程:}
	面向相变诱导体积膨胀导致的粒子稀疏与邻域退化,建立分裂/合并触
	发指标与重构流程,使粒子分辨率与邻域支撑在体积剧变过程中保持可控;
	重点约束相界面附近与相变活跃区的粒子表示质量,避免邻域退化引起
	界面通量计算失真并破坏相变闭合。
	\item[] \textbf{(2)分裂/合并后的守恒变量重分配与一致修正:}
	构建分裂/合并与重采样后的质量、动量与能量变量重分配方法,保证
	局部守恒并与界面耦合与相变闭合的离散形式一致;
	通过一致重映射与必要的修正环节,减少重构引入的隐性质量/能量
	误差与压力扰动,避免潜热重复计量在重构过程中累积。
	\item[] \textbf{(3)多时间尺度推进的组织与界面交换对齐:}
	针对声学CFL、热扩散与相变更新带来的多时间尺度约束,研究分层时间
	推进组织方式,建立气相子步与液相大步耦合推进的界面交换对齐策略,
	明确界面交换、相变更新与粒子重构的时间层协同安排,避免时序错配
	导致的相变速率波动与界面耦合不稳;并在体积剧变相变算例中检验
	推进效率与稳定裕度。
\end{enumerate}

\textbf{\textcolor{B}{研究内容间支撑关系:}}
三项研究以守恒一致更新流程为主线,两两耦合、相互约束,其关系如图
\ref{研究内容}所示。
研究内容1提供稳定的界面压力-速度传递与几何输入,支撑研究内容2的热通
量评估与相变闭合。
相变引起的质量/能量变化与体积膨胀将反向影响界面传递精度并触发粒子
自适应需求,研究内容3通过守恒重分配与时间层对齐,约束并保障研究内
容1与2在体积剧变条件下的稳定与一致。
%		
\section{拟解决的关键科学问题}
本项目拟解决的关键科学问题如下:

(1)\textbf{\textcolor{B}{相界面物性跃变处压力-速度信息
		的稳定传递与界面耦合一致离散}}。
分离式两相SPH中,液相与气相推进策略不同,界面处需通过压力
边界与速度约束实现双向耦合。
在高密度比与邻域退化条件下,核支撑域截断与粒子扰动易放大界面
离散误差,诱发压力振荡、寄生流与相间掺混,并导致稳定域收缩。
\textbf{\textcolor{B}{关键在于界面边界条件的离散构造与
		交换方式未与两相推进过程一致匹配}},界面误差易进入压力-速度
更新并被持续放大。
需要在受限核支撑与界面几何约束下,统一压力边界重构、速度约束
施加与稳定化强度选取,使界面传递与推进算子保持同构一致。
因此,本项目将构建与推进算子一致的界面边界离散与交换机制,
形成可控的压力-速度稳定传递策略,为高密度比两相计算提供稳健
界面耦合框架。		

(2)\textbf{\textcolor{B}{相变闭合与粒子自适应在统一
		离散框架下的守恒一致实现}}。
气液相变由界面热通量驱动,质量交换与潜热更新会反馈影响压力
求解与速度更新;相变诱导的体积膨胀又会导致邻域退化并触发
粒子重构需求。
相变闭合、潜热守恒更新与粒子分裂/合并若分步实施,守恒误差
与时间错配易累积,表现为能量漂移、潜热重复计入或相变速率波
动,并削弱整体稳定性。
\textbf{\textcolor{B}{关键在于相变更新与粒子拓扑重构未
		纳入统一守恒更新流程并实现时间对齐}},局部误差会反向扰动
界面耦合稳定性。
需要将热通量驱动的相变闭合、潜热守恒更新、分裂/合并后的
变量重分配与多时间尺度界面交换纳入同一更新链路,避免守恒
与时序错配的累积。
因此,本项目将构建相变闭合与潜热守恒更新方法,并与粒子自
适应重构及多时间尺度推进协同设计,实现体积剧变条件下相变
两相流的稳定可靠计算。				
%
\section{拟采取的研究方案}
\subsection{技术路线}
验证算例:

\textbf{\textcolor{R}{Yang\cite{yang2017smoothed}:}}
Stefan problem、Evaporation of a static drop、Evaporation of a drop 
impacting a hot surface

\textbf{\textcolor{R}{Huang\cite{he2022stable}:}}
This current research integrates SPH models of heat transfer, fluid 
dynamics, and phase transitions. 传热、传质、相变的模拟	
Stefan problem, natural convection problems involving phase transitions,
对比实验或者有限元分析(主要是COMSOL的结果)

\textbf{\textcolor{R}{Duan\cite{duan2020incompressible}:}}:沸腾相变
气液相变涉及到(1)高密度比和体积的(2)急剧膨胀问题。
开发汽化或沸腾相变模型可以显著拓宽颗粒方法的应用范围。
验证算例:rising bubble problem,Stefan problem,sucking problem,
film boiling problem

\subsection{研究方法}

\subsection{关键技术}

\section{本研究特色创新之处}
\textbf{\textcolor{R}{在这一部分要帮评审人把要复制的话写出来}}